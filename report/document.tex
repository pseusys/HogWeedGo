\section{Введение}
\tab
В настоящий момент становится особенно актуальной проблема распространения потенциально опасных для человека видов растений.
При этом под потенциально опасными следует понимать как те растения, которые представляют прямую угрозу жизни и здоровью человека, так и те, которые негативно влияют на сельскохозяйственную, промышленную и иные сферы деятельности человека\cite{alien-plants}.
Отчасти актуальность этой проблемы объясняется расширением пригодных для произрастания таких растений территорий\cite{invasive-plants, poisonous-ivy}.

\nwln
В европейской части России одним из самых опасных растений является борщевик Сосновского\cite{heracleum}, поэтому в данной работе контроль над распространением потенциально опасных растений рассматривается на его примере.
Согласно данным из открытой базы данных РИВР\cite{rivr} на территории Санкт-Петербурга и Ленинградской области насчитывается более 350 очагов распространения Heracleum sosnowskyi.
По данным, опубликованным в журнале Esquire\cite{esquire} борщевик занимает до 15\% природных ландшафтов европейской части России.

\nwln
На сегодняшний день не существует специализированного автоматизированного компьютерного решения для контроля над распространением потенциально опасных видов растений, которое, помимо наблюдения и каталогизации, включало бы в себя взаимодействие с группами специалистов, способными устранить и остановить очаги распространения.

\nwln
\textbf{Целью данной работы} стала разработка такой системы, позволяющей решить проблему распространения потенциально опасных для человека видов растений и контроля их популяции.
Такая система должна позволить широким массам добровольцев осуществлять наблюдение за потенциально опасными видами растений, а также давать группе экологов возможность осуществлять модерацию и анализ собранных данных.

\nwln
\textbf{Для достижения данной цели были сформулированы следующие задачи}:
\begin{enumerate}[topsep=0pt, parsep=0pt, itemsep=0pt, wide=0.5cm]
	\item Разработка архитектуры, описание ее компонентов и путей передачи данных.
	\item Создание сервера для сбора и хранения данных, а также системы анализа и модерирования наблюдений с web-интерфейсом.
	\item Создание кроссплатформенного мобильного и web приложения, позволяющего добровольцем заносить наблюдения в систему.
\end{enumerate}

\nwln
\textbf{Объектом данного исследования} являются особенности создания кроссплатформенной системы, реализующей клиент-серверную архитектуру и позволяющей решить проблему распространения вредоносных растений.

\nwln
\textbf{Предметом данного исследования} является распространение опасных для человека видов растений, контроль их популяции.

\nwln
\textbf{Практическая ценность данной работы} заключается в возможности контроля над распространением потенциально опасных видов растений (в частности, Борщевика Сосновского) на территории Санкт-Петербурга и Ленинградской области.

\section{Обзор предметной области}

\subsection{Понятие контроля}

\tab
В словаре\cite{control} указано, что слово \textquote{контроль} означает проверку и наблюдение с целью проверки.
В рамках данной работы под контролем над распространением потенциально опасных для человека видов растений предлагается понимать также возможность принятия необходимых мер для устранения негативных последствий распространения этих растений.
Предполагается, что контроль над такими растениями будет поручен группе учёных-экологов, в обязанности которых будет входить работа с системой.

\nwln
Для осуществления контроля (по данному выше определению) необходимо хранение информации об очагах распространения опасных растений, которая включает в себя фотографии растения, его географические координаты и краткое описание, данное обнаружившим его человеком, самого растения, а также места его произрастания.
Такой набор данных в данной работе будет называться \linkterm{report}{отчёт}.
В случае наблюдения за несколькими видами потенциально опасных растений каждый отчёт также должен содержать информацию о видовой принадлежности обнаруженного растения.
В следствии того, что эти сведения со временем могут терять актуальность или же требовать модерации, в отчёте необходимо также хранить информацию о статусе данных и дате совершения наблюдения.
Предлагается использование трёх статусов: \textquote{необработано}, \textquote{актуально}, \textquote{неактуально}; в зависимости от необходимых мер контроля можно добавить и другие.

\nwln
Для осуществления поиска и анализа данных необходима функция сортировки и поиска конкретного наблюдения по нескольким признакам. Такая функция в данной работе будет называться \linkterm{processing}{обработка данных}. \\

\subsection{Источник информации}

\tab
В качестве источника информации и средства наблюдения предлагается использовать неограниченное количество волонтёров, заинтересованных в решении экологических проблем.
Эти люди в данной работе будут называться \linkterm{users}{пользователями} системы.
Такое решение позволит получать большое количество актуальной информации, но в следствие отсутствия профессиональной подготовки волонтёров точность такой информации снизится.
Также возможно возникновение случаев необходимости модерации присланных волонтёрами данных.

\nwln
В качестве меры модерации предлагается необходимость регистрации волонтёров при помощи подтверждённого адреса электронной почты (в последствии возможно добавление других методов авторизации), а также наличие возможности редактирования и удаления всех загруженных данных.
Кроме того, для предоставления возможности связи с волонтёрами, в каждый отчёт необходимо включить информацию о пользователе, совершившем наблюдение, описанное в отчёте.
Люди, в обязанности которых будет входить модерация отчётов и взаимодействие с пользователями в данной работе будут называться \linkterm{admins}{администраторами} системы.

\nwln
В качестве меры повышения точности присланных волонтёрами данных предлагается создание автоматической системы распознавания определённых видов растений на фотографиях, а также ручная проверка присланных пользователями данных.
Люди, в обязанности которых будет входить проверка и обработка присланных данных в данной работе будут называться \linkterm{specialists}{специалистами} системы. \\

\subsection{Доступ к данным}

\tab
Для охвата более широкой группы добровольцев и их удобства необходима возможность отложенной отправки данных в случае, если пользователь находится вне зоны действия сети.
Этого возможно добиться при помощи создания мобильных приложений для операционных систем iOS и Android.
Также необходимо наличие вебинтерфейса отправки данных для пользователей, не желающих устанавливать приложение.
Кроме того желательно предоставление возможности интеграции системы со сторонними клиентскими приложениями путём создания документированного API для отправки данных в систему.

\nwln
Для осуществления модерации и обработки данных требуется создание сервисного вебинтерфейса, который далее в данной работе будет называться \linkterm{servicesite}{сервисный сайт} системы.
Также для удобства использования собранных данных в полевых условиях (например, во время поиска и обезвреживания растения), а также сбора статистики необходимо наличие способа выгрузки и сохранения обработанных данных в формате \textquote{CSV}.
Также желательно наличие возможности демонстрации статистики собранных данных на сторонних сайтах путём создания документированного API для демонстрации данных, загруженных в систему. \\

\subsection{Существующие решения}

\tab
Несмотря на то, что системы, решающей именно эту проблему пока не существует, можно найти множество систем с похожей функциональностью.
Для того, чтобы определить, нельзя ли вместо создания нового использовать готовое решение или его фрагменты, было произведено сравнение систем с аналогичной или похожей функциональностью по нескольким критериям.

\nwln
Поиск аналогов производился среди систем, которые используются экологами для фиксации месторасположения борщевика в настоящий момент, а также по ключевым словам “фотофиксация растений”, “поиск и фотофиксация”, а также “составление пользовательских карт”.
В качестве аналогов были выбраны сервисы, позволяющие отмечать что-либо на карте, производить поиск растений или же просто собирать и обрабатывать пользовательские данные.

\begin{enumerate}[topsep=0pt, parsep=0pt, itemsep=0pt, wide=0.5cm]
	\item iNaturalist \\
	iNaturalist\cite{inaturalist} - сервис для наблюдения за живыми организмами, разработанный Калифорнийской академией наук совместно с National Geographic.
	Сервис позволяет глобальному сообществу волонтёров наблюдать за живыми организмами и присылать собранную информацию о них на сайт.
	Такие отчёты находятся в общем доступе и могут быть загружены с официального сайта.
	\item РИВР \\
	РИВР\cite{rivr} (распространение инвазивных видов растений) - открытая база данных, разработанная и поддерживаемая ИБ Коми НЦ УрО РАН.
	Посвящена исключительно Борщевику Сосновского и его распространению на территории России.
	Позволяет пользователям отправлять данные о найденных местах произрастания растения, которые на данный момент доступны только для просмотра на сайте.
	\item Google MyMaps \\
	Google MyMaps\cite{googlemymaps} - это сервис для создания пользовательских карт на основе сервиса Google Maps.
	Позволяет создавать карты с набором пользовательских отметок и их описанием, доступ к таким картам осуществляется так же, как к файлам, хранящимся на Google Drive.
	Также позволяет единовременно импортировать данные их .csv файла или из Google SpreadSheets.
	\item Plant Finder \\
	Plant Finder\cite{plantfinder} - сервис, созданный для документации и
	учета мест произрастания и хранения растений. Сервис
	предоставляет пользователям возможность отмечать на
	карте места собранных ими коллекций растений. Также
	для пользователей доступна возможность создания
	частных карт с пользовательским набором растений. На
	таких частных картах есть возможность открытия
	публичного доступа к отметкам и данным.
	\item Flora Incognita \\
	Flora Incognita\cite{floraincognita} - приложение, позволяющее
	идентифицировать и сохранять данные о найденных
	растениях. Оно разработано совместно с Technische
	Universität Ilmenau и Max-Planck-Institut für Biogeochemie.
	Также в приложении доступно описание каждого
	растения, которое оно может идентифицировать.
\end{enumerate}

\nwln
Для сравнения аналогов были использованы следующие критерии, основанные на анализе предметной области:

\begin{itemize}[topsep=0pt, parsep=0pt, itemsep=0pt, wide=0.5cm]
	\item Полнота данных -- возможность хранения всех необходимых для отчётов данных, а именно: фотографию и краткое описание растения, описание места его произрастания, географические координаты, дату и время обнаружения, данные о пользователе, который его произвёл, и текущий статус отчета.
	\item Обработка данных -- возможность проверки принадлежности растений к отслеживаемым видам, а также обработки данных.
	\item Мобильный интерфейс -- возможность отправки отчётов в систему как при помощи приложений для операционных систем iOS и Android так и при помощи вебинтерфейса, а также существование API для сбора данных.
	\item Доступ к данным -- возможность выгрузки и сохранения данных в формате “CSV”, наличие демонстрации статистики собранных данных на сайте, а также наличие API для демонстрации данных.
	\item Модерация данных -- возможность модерирования данных, управления аккаунтами пользователей, а также связи с пользователями при помощи подтверждённых при регистрации методов.
\end{itemize}

\nwln
Результаты сравнения существующих аналогов приведены в таблице 1.
\tableone

\nwln
Из анализа соответствия существующих аналогов необходимым критериям становится очевидно, что на данный момент не существует системы, предоставляющей требуемую функциональность в полном объёме.
Вследствие этого было принято решение о необходимости разработки такой системы.

\section{Формулировка требований и постановка задачи}

\#\# TODO after this point!!

\subsection{Формулировка требований}

Этот сервис должен удовлетворять требованиям, описанным в пункте “Критерии сравнения аналогов”.
Сервис должен обладать следующей функциональностью:
\begin{itemize}[topsep=0pt, parsep=0pt, itemsep=0pt, wide=0.5cm]
	\item У группы пользователей-волонтеров должна быть возможность отправлять данные о местонахождении встреченного ими потенциально опасного растения.
	\item Для удобства отправки данных необходимо наличие как web-интерфейса, так и мобильного интерфейса (приложения) для платформ Android и iOS, позволяющего откладывать отправку в случае нахождения устройства вне зоны действия сети.
	\item У группы экологов-администраторов должна быть возможность хранить, обрабатывать, а так же модерировать (в случае их неподобающего содержания) полученные данные.
\end{itemize}

\subsection{Постановка задачи}

\subsection{Сценарии использования}

\section{Архитектура решения}

\subsection{Используемые технологии}

\subsection{Модель данных}

\section{Описание решения}

\section{Заключение}
