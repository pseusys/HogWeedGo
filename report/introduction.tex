% !TeX spellcheck = russian-aot

\newcont{Введение}

\tab
В настоящий момент становится особенно актуальной проблема распространения потенциально опасных для человека видов растений.
При этом под потенциально опасными следует понимать как те растения, которые представляют прямую угрозу жизни и здоровью человека, так и те, которые негативно влияют на сельскохозяйственную, промышленную и иные сферы деятельности человека\cite{alien-plants}.
Отчасти актуальность этой проблемы объясняется расширением пригодных для произрастания таких растений территорий\cite{invasive-plants}.

\nwln
В европейской части России одним из самых опасных растений является борщевик Сосновского\cite{heracleum}, поэтому в данной работе наблюдение за распространением потенциально опасных растений рассматривается на его примере.
Согласно данным из открытой базы данных РИВР\cite{rivr} на территории Санкт-Петербурга и Ленинградской области насчитывается более 350 очагов распространения Heracleum sosnowskyi.
По данным, опубликованным в журнале Esquire\cite{esquire} борщевик занимает до 15\% природных ландшафтов европейской части России.

\nwln
На сегодняшний день не существует специализированного автоматизированного компьютерного решения для контроля над распространением потенциально опасных видов растений, которое, помимо наблюдения и каталогизации, включало бы в себя взаимодействие с группами специалистов, способными устранить и остановить очаги распространения.

\nwln
\textbf{Целью данной работы} стала разработка такой системы, позволяющей решить проблему распространения потенциально опасных для человека видов растений и контроля их популяции.
Такая система должна позволить широким массам добровольцев осуществлять наблюдение за потенциально опасными видами растений, а также давать группе экологов возможность осуществлять модерацию и анализ собранных данных.

\nwln
\textbf{Для достижения данной цели были сформулированы следующие задачи}:
\begin{enumerate}[topsep=0pt, parsep=0pt, itemsep=0pt, wide=0.5cm]
	\item Разработка архитектуры, описание ее компонентов и путей передачи данных.
	\item Создание сервера для сбора и хранения данных, а также системы анализа и модерирования наблюдений с web-интерфейсом.
	\item Создание кроссплатформенного мобильного и web приложения, позволяющего добровольцем заносить наблюдения в систему.
\end{enumerate}

\nwln
\textbf{Объектом данного исследования} являются особенности создания кроссплатформенной системы, реализующей клиент-серверную архитектуру и позволяющей решить проблему распространения вредоносных растений.

\nwln
\textbf{Предметом данного исследования} является распространение опасных для человека видов растений, контроль их популяции.

\nwln
\textbf{Практическая ценность данной работы} заключается в возможности контроля над распространением потенциально опасных видов растений (в частности, Борщевика Сосновского) на территории Санкт-Петербурга и Ленинградской области. 
