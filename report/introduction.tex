% !TeX spellcheck = russian-aot

\newcont{Введение}

\tab
В настоящий момент становится особенно актуальной проблема распространения потенциально опасных для человека видов растений.
При этом под потенциально опасными следует понимать как те растения, которые представляют прямую угрозу жизни и здоровью человека, так и те, которые негативно влияют на сельскохозяйственную, промышленную и иные сферы деятельности человека\cite{alien-plants}.
Отчасти актуальность этой проблемы объясняется расширением пригодных для произрастания таких растений территорий\cite{invasive-plants}.

\tab
В европейской части России одним из самых опасных растений является борщевик Сосновского\cite{heracleum}, поэтому в данной работе наблюдение за распространением потенциально опасных растений рассматривается на его примере.
Согласно данным из открытой базы данных РИВР\cite{rivr} на территории Санкт-Петербурга и Ленинградской области насчитывается более 350 очагов распространения \la{Heracleum sosnowskyi}.
По данным, опубликованным в журнале \en{Esquire}\cite{esquire} борщевик занимает до 15\% природных ландшафтов европейской части России.

\tab
На сегодняшний день не существует специализированного автоматизированного компьютерного решения для контроля над распространением потенциально опасных видов растений, которое, помимо наблюдения и каталогизации, включало бы в себя взаимодействие с группами специалистов, способными устранить и остановить очаги распространения.

\tab
\textbf{Целью данной работы} стала разработка такой системы, позволяющей решить проблему распространения потенциально опасных для человека видов растений и контроля их популяции.
Такая система должна позволить широким массам добровольцев осуществлять наблюдение за потенциально опасными видами растений, а также давать группе экологов возможность осуществлять модерацию и анализ собранных данных.

\tab
\textbf{Для достижения данной цели были сформулированы следующие задачи}:
\begin{enumerate}
	\item Разработка архитектуры, описание ее компонентов и путей передачи данных.
	\item Создание серверного приложения для сбора и хранения данных, а также системы анализа и модерирования наблюдений с web-интерфейсом.
	\item Создание кроссплатформенного мобильного и web приложения, позволяющего добровольцем заносить наблюдения в систему.
\end{enumerate}

\tab
\textbf{Объектом данного исследования} является автоматизированный учёт и хранение данных о распространение вредоносных растений.

\tab
\textbf{Предметом данного исследования} является удобство и скорость отправки и получения данных.
Под удобством подразумевается длина последовательности действий, которые необходимо произвести пользователю для выполнения задачи.
Минимизация длины этой последовательности считается повышением удобства.

\tab
\textbf{Практическая ценность данной работы} заключается в возможности контроля над распространением потенциально опасных видов растений (в частности, борщевика Сосновского).
При необходимости с помощью описанного ПО можно проводить наблюдения за распространением любых видов растений.

\tab
По теме работы существует публикация\cite{article}, описывающая разработку архитектуры системы.
