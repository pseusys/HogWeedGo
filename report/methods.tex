% !TeX spellcheck = russian-aot

\section{Выбор метода решения}

\tab
В качестве источника информации и средства наблюдения предлагается использовать неограниченное количество волонтёров, заинтересованных в решении экологических проблем.
Эти люди в данной работе будут называться \linkterm{users}{пользователями} системы.
Такое решение позволит получать большое количество актуальной информации, но в следствие отсутствия профессиональной подготовки волонтёров точность такой информации снизится.
Также возможно возникновение случаев необходимости модерации присланных волонтёрами данных.

\tab
В качестве меры модерации предлагается необходимость регистрации волонтёров при помощи подтверждённого адреса электронной почты (в последствии возможно добавление других методов авторизации), а также наличие возможности редактирования и удаления всех загруженных данных.
Кроме того, для предоставления возможности связи с волонтёрами, в каждый отчёт необходимо включить информацию о пользователе, совершившем наблюдение, описанное в отчёте.
Люди, в обязанности которых будет входить модерация отчётов и взаимодействие с пользователями в данной работе будут называться \linkterm{admins}{администраторами} системы.

\tab
В качестве меры повышения точности присланных волонтёрами данных предлагается создание автоматической системы распознавания определённых видов растений на фотографиях, а также ручная проверка присланных пользователями данных.
Желательно повысить точность предварительной автоматической фильтрации фотографий до 90\% или выше, при этом размер модели нейронной сети не должен превышать 10 mb, что позволит использование её в клиентских приложениях.
Люди, в обязанности которых будет входить проверка и работа с присланными данными в данной работе будут называться \linkterm{specialists}{специалистами} системы.

\tab
Для охвата более широкой группы добровольцев и их удобства необходима возможность отложенной отправки данных в случае, если пользователь находится вне зоны действия сети.
Этого возможно добиться при помощи создания мобильных приложений для операционных систем iOS и \en{Android}.
Также необходимо наличие web интерфейса отправки данных для пользователей, не желающих устанавливать приложение.
Кроме того желательно предоставление возможности интеграции системы со сторонними клиентскими приложениями путём создания документированного API для отправки данных в систему.

\tab
Для осуществления модерации и работы с данными требуется создание сервисного web интерфейса, который далее в данной работе будет называться \linkterm{servicesite}{сервисный сайт} системы.
Также для удобства использования собранных данных в полевых условиях (например, во время поиска и обезвреживания растения), а также сбора статистики необходимо наличие способа выгрузки и сохранения обработанных данных в формате \textquote{CSV}.
Также желательно наличие возможности демонстрации статистики собранных данных на сторонних сайтах путём создания документированного API для демонстрации данных, загруженных в систему.
