% !TeX spellcheck = russian-aot

\section{Обзор предметной области}

\subsection{Понятие наблюдения}

\tab
В рамках данной работы под наблюдением за распространением потенциально опасных для человека видов растений предлагается понимать сбор, каталогизацию и хранение данных о них.
Предполагается, что наблюдение за такими растениями будет поручено группе учёных-экологов, в обязанности которых будет входить работа с системой.

\nwln
Необходимая для осуществления наблюдения информации об очагах распространения опасных растений включает в себя фотографии растения, его географические координаты и краткое описание, данное обнаружившим его человеком, самого растения, а также места его произрастания.
Такой набор данных в данной работе будет называться \linkterm{report}{отчёт}.
Функция сортировки и поиска конкретного отчёта по нескольким признакам в данной работе будет называться \linkterm{processing}{работа с данными}.

\nwln
В случае наблюдения за несколькими видами потенциально опасных растений каждый отчёт также должен содержать информацию о видовой принадлежности обнаруженного растения.
В следствии того, что эти сведения со временем могут терять актуальность или же требовать модерации, в отчёте необходимо также хранить информацию о статусе данных и дате совершения наблюдения.
Предлагается использование трёх статусов: \textquote{необработано}, \textquote{актуально}, \textquote{неактуально}; в зависимости от ситуации и цели наблюдения возможно добавление и других. \\

\subsection{Существующие решения}

\tab
Несмотря на то, что системы, решающей именно эту проблему пока не существует, можно найти множество систем с похожей функциональностью.
Для того, чтобы определить, нельзя ли вместо создания нового использовать готовое решение или его фрагменты, было произведено сравнение систем с аналогичной или похожей функциональностью по нескольким критериям.

\nwln
Поиск аналогов производился среди систем, которые в настоящий момент используются экологами для фиксации месторасположения борщевика, а также в интернете по ключевым словам \textquote{фотофиксация растений}, \textquote{поиск и фотофиксация}, а также \textquote{составление пользовательских карт}.
В качестве аналогов были выбраны сервисы, позволяющие отмечать что-либо на карте, производить поиск растений или же просто собирать и обрабатывать пользовательские данные.

\begin{enumerate}[topsep=0pt, parsep=0pt, itemsep=0pt, wide=0.5cm]
	\item \en{iNaturalist} \\
	\en{iNaturalist}\cite{inaturalist} - сервис для наблюдения за живыми организмами, разработанный Калифорнийской академией наук совместно с \en{National Geographic}.
	Сервис позволяет глобальному сообществу волонтёров наблюдать за живыми организмами и присылать собранную информацию о них на сайт.
	Такие отчёты находятся в общем доступе и могут быть загружены с официального сайта.
	\item РИВР \\
	РИВР\cite{rivr} (распространение инвазивных видов растений) - открытая база данных, разработанная и поддерживаемая ИБ Коми НЦ УрО РАН.
	Посвящена исключительно Борщевику Сосновского и его распространению на территории России.
	Позволяет пользователям отправлять данные о найденных местах произрастания растения, которые на данный момент доступны только для просмотра на сайте.
	\item \en{Google MyMaps} \\
	\en{Google MyMaps}\cite{googlemymaps} - это сервис для создания пользовательских карт на основе сервиса \en{Google Maps}.
	Позволяет создавать карты с набором пользовательских отметок и их описанием, доступ к таким картам осуществляется так же, как к файлам, хранящимся на \en{Google Drive}.
	Также позволяет единовременно импортировать данные их .csv файла или из \en{Google SpreadSheets}.
	\item \en{Plant Finder} \\
	\en{Plant Finder}\cite{plantfinder} - сервис, созданный для документации и учёта мест произрастания и хранения растений.
	Сервис предоставляет пользователям возможность отмечать на карте места собранных ими коллекций растений.
	Также для пользователей доступна возможность создания частных карт с пользовательским набором растений.
	На таких частных картах есть возможность открытия публичного доступа к отметкам и данным.
	\item \la{Flora Incognita} \\
	\la{Flora Incognita}\cite{floraincognita} - приложение, позволяющее идентифицировать и сохранять данные о найденных растениях.
	Оно разработано совместно с \de{Technische Universität Ilmenau} и \de{Max-Planck-Institut für Biogeochemie}.
	Также в приложении доступно описание каждого растения, которое оно может идентифицировать.
\end{enumerate}

\nwln
Для сравнения аналогов были использованы следующие критерии:

\begin{itemize}[topsep=0pt, parsep=0pt, itemsep=0pt, wide=0.5cm]
	\item Полнота данных -- возможность хранения всех необходимых для отчётов данных, а именно: фотографию и краткое описание растения и места его произрастания, географические координаты, дату и время обнаружения, данные о пользователе, который его произвёл, и текущий статус отчёта.
	\item Обработка данных -- возможность проверки принадлежности растений к отслеживаемым видам, а также работы с данными.
	\item Мобильный интерфейс -- возможность отправки отчётов в систему как при помощи приложений для операционных систем iOS и \en{Android} так и при помощи web интерфейса, а также существование API для сбора данных.
	\item Доступ к данным -- возможность выгрузки и сохранения данных в формате “CSV”, наличие демонстрации статистики собранных данных на сайте, а также наличие API для демонстрации данных.
	\item Модерация данных -- возможность модерирования данных, управления аккаунтами пользователей, а также связи с пользователями при помощи подтверждённых при регистрации методов.
\end{itemize}

\nwln
Результаты сравнения существующих аналогов приведены в таблице 1. \\
\tableone

\nwln
Из анализа соответствия существующих аналогов необходимым критериям становится очевидно, что на данный момент не существует системы, предоставляющей требуемую функциональность в полном объёме.
Вследствие этого было принято решение о необходимости разработки такой системы.
