% !TeX spellcheck = russian-aot

\section{Проектирование решения}

\tab
Для реализации системы была выбрана клиент-серверная архитектура.

\tab
Данные собираются на сервере и сохраняются в базе данных при помощи серверной части ПО.
Она позволяет специалистам и администраторам системы работать с данными при помощи специализированного сервисного сайта.
Она же обеспечивает функционирование API системы, часть методов которого не требуют аутентификации и позволяют получать обработанные, отсортированные и отфильтрованные данные, а часть - требует и позволяет отправлять новые данные в систему.

\tab
Получить доступ к этим данным можно при помощи приложения-клиента или любого сайта, который сделает запрос к API системы.
Приложение клиент также позволяет пользователям управлять своим аккаунтом, собирать данные в отчёт и отправлять его в систему.

\tab
Схема архитектуры представлена на рисунке 1.
\pictureone

\subsection{Сценарии использования}

\tab
Были разработаны следующие сценарии использования:

\subsubsection{Сценарии использования клиентского приложения}

\tab
Действующее лицо: пользователь.

\begin{itemize}[topsep=0pt, parsep=0pt, itemsep=0pt, leftmargin=*, labelindent=0.5cm]
	\item \textit{Просмотр опубликованной отсортированной информации о наблюдениях} \\
	Основной сценарий:
	\begin{itemize}[topsep=0pt, parsep=0pt, itemsep=0pt, leftmargin=*, labelindent=0.5cm]
		\item Пользователь открывает клиентское приложение.
		\item Пользователь нажимает на один из указателей на карте, ему демонстрируются подробные сведения об отчёте и комментарии к нему.
		\item По желанию пользователь нажимает на фотографии, прикреплённые к отчёту и просматривает их увеличенную версию.
	\end{itemize}
	Результат: Пользователь ознакомился как с общей статистикой распространения потенциально опасных растений на карте, так и с заинтересовавшими его конкретными отчётами.
	
	\item \textit{Просмотр информации о системе, её задачах и краткого руководства по использованию} \\
	Основной сценарий:
	\begin{itemize}[topsep=0pt, parsep=0pt, itemsep=0pt, leftmargin=*, labelindent=0.5cm]
		\item На основной странице приложения пользователь открывает меню навигации и выбирает пункт \textquote{\en{About}}, ему демонстрируется информация о системе, её задачах и краткое руководство по использованию.
	\end{itemize}
	Результат: Пользователь посмотрел данные со всей информацией по операциям конкретного изделия.
	
	\item \textit{Авторизация в системе} \\
	Условие: На основной странице приложения пользователь открывает меню навигации и выбирает пункт \textquote{\en{Account}} ИЛИ на основной странице приложения нажимает на кнопку справа внизу с символом \textquote{+} ИЛИ на странице просмотра отчёта нажимает на ссылку \textquote{\en{Sign in}}. \\
	Основной сценарий:
	\begin{itemize}[topsep=0pt, parsep=0pt, itemsep=0pt, leftmargin=*, labelindent=0.5cm]
		\item Открывается страница авторизации пользователя.
		\item Пользователь нажимает кнопку \textquote{\en{Already have account?}}.
		\item Пользователь вводит свой адрес электронной почты в поле ввода адреса электронной почты.
		\item Пользователь вводит свой пароль в поле ввода пароля.
		\item Пользователь вводит свой пароль в поле ввода подтверждения пароля.
		\item Пользователь подтверждает согласие на сбор и обработку персональных данных, поставив флаг \textquote{\en{Agree to terms and conditions}}.
		\item Пользователь нажимает кнопку \textquote{\en{Submit}}.
		\item Пользователь подтверждает указанный адрес электронной почты, в открывшемся диалоговом окне он вводит восьмизначный код, отправленный на его электронную почту.
		\item Пользователь нажимает кнопку \textquote{\en{Submit}}.
		\item Диалоговое окно закрывается.
		\item В нижнем правом углу появляется сообщение о статусе авторизации и, в случае её успеха, открывается главная страница приложения.
	\end{itemize}
	Альтернативный сценарий (условие - наличие у пользователя аккаунта):
	\begin{itemize}[topsep=0pt, parsep=0pt, itemsep=0pt, leftmargin=*, labelindent=0.5cm]
		\item Открывается страница авторизации пользователя.
		\item Пользователь вводит свой адрес электронной почты в поле ввода адреса электронной почты.
		\item Пользователь вводит свой пароль в поле ввода пароля
		\item Пользователь нажимает кнопку \textquote{\en{Submit}}.
		\item В нижнем правом углу появляется сообщение о статусе авторизации и, в случае её успеха, открывается главная страница приложения.
	\end{itemize}
	Результат: Пользователь авторизован в системе и может выполнять требующие авторизации сценарии.
	
	\item \textit{Отправка отчёта о наблюдении потенциально опасного растения} \\
	Условие: Пользователь авторизован в системе. \\
	Основной сценарий:
	\begin{itemize}[topsep=0pt, parsep=0pt, itemsep=0pt, leftmargin=*, labelindent=0.5cm]
		\item На основной странице приложения нажимает на кнопку справа внизу с символом \textquote{+}.
		\item Снизу экрана открывается форма отправки отчёта.
		\item Пользователь заполняет поле описания находки и, по желанию, описания местоположения находки.
		\item Пользователь выбирает местоположение своей находки на карте.
		\item Пользователь заполняет поле даты и времени обнаружения растения.
		\item По желанию пользователь загружает фотографии находки. Фотографии проверяются системой автоматической проверки изображений и пользователь ознакамливается с результатами проверки справа от поля для загрузки фотографий.
		\item По желанию пользователь нажимает на фотографии, прикреплённые к отчёту и просматривает их увеличенную версию.
		\item Пользователь нажимает кнопку \textquote{\en{Send}}.
		\item Форма отправки отчёта закрывается.
		\item В нижнем правом углу появляется сообщение о статусе отправки и, в случае успеха, на карту добавляется новая метка в указанном месте.
	\end{itemize}
	Результат: Пользователь отправил отчёт о наблюдении потенциально опасного растения, его фотографии были предварительно проверены и данные загружены в систему.
	
	\item \textit{Написание комментария к отчёту} \\
	Условие: Пользователь авторизован в системе И на основной странице приложения пользователь нажимает на один из указателей на карте. \\
	Основной сценарий:
	\begin{itemize}[topsep=0pt, parsep=0pt, itemsep=0pt, leftmargin=*, labelindent=0.5cm]
		\item Пользователь вводит свой комментарий к отчёту в поле для ввода комментария и нажимает кнопку с символом \textquote{>}.
		\item Поле для ввода комментария очищается.
		\item В нижнем правом углу появляется сообщение о статусе отправки комментария.
	\end{itemize}
	Результат: Пользователь оставил комментарий, дополнив информацию, изложенную в отчёте.
	
	\item \textit{Редактирование профиля} \\
	Условие: Пользователь авторизован в системе. \\
	Основной сценарий:
	\begin{itemize}[topsep=0pt, parsep=0pt, itemsep=0pt, leftmargin=*, labelindent=0.5cm]
		\item На основной странице приложения пользователь открывает меню навигации и выбирает пункт \textquote{\en{Account}}.
		\item По желанию пользователь выбирает новую фотографию профиля.
		\item По желанию пользователь вводит новое имя и нажимает \textquote{\en{Send}}.
		\item По желанию пользователь нажимает на кнопку \textquote{Адрес электронной почты}, вводит новый адрес электронной почты, нажимает на кнопку \en{Prove} и вводит код её подтверждения в открывшемся диалоговом окне, после чего нажимает на кнопку \en{Submit}.
		\item По желанию пользователь нажимает на кнопку \textquote{Изменить пароль} и вводит новый пароль и его подтверждение в открывшемся диалоговом окне, после чего нажимает на кнопку \en{Submit}.
		\item По желанию пользователь нажимает на кнопку \textquote{Выйти} ИЛИ \textquote{Удалить аккаунт}.
		\item В нижнем правом углу появляется сообщение о статусе обновления профиля.
		\item В случае выхода из системы открывается главная страница приложения.
	\end{itemize}
	Результат: Пользователь оставил комментарий, дополнив информацию, изложенную в отчёте.
\end{itemize}

\subsubsection{Сценарии использования сервисного сайта}

\begin{itemize}[topsep=0pt, parsep=0pt, itemsep=0pt, leftmargin=*, labelindent=0.5cm]
	\item \textit{Вход на сервисный сайт} \\
	Действующее лицо: специалист ИЛИ администратор. \\
	Основной сценарий:
	\begin{itemize}[topsep=0pt, parsep=0pt, itemsep=0pt, leftmargin=*, labelindent=0.5cm]
		\item Специалист ИЛИ администратор вводит свой адрес электронной почты в поле ввода адреса электронной почты.
		\item Специалист ИЛИ администратор вводит свой пароль в поле ввода пароля.
	\end{itemize}
	Результат: Специалист ИЛИ администратор авторизован в системе и может выполнять требующие авторизации сценарии.
\end{itemize}

\begin{itemize}[topsep=0pt, parsep=0pt, itemsep=0pt, leftmargin=*, labelindent=0.5cm]
	\item \textit{Поиск отчета по карте} \\
	Действующее лицо: специалист ИЛИ администратор. \\
	Основной сценарий:
	\begin{itemize}[topsep=0pt, parsep=0pt, itemsep=0pt, leftmargin=*, labelindent=0.5cm]
		\item Специалист ИЛИ администратор находит интересующий его маркер отчёта на карте.
		\item Специалист ИЛИ администратор нажимает на него.
		\item Открывается страница редактирования отчёта.
	\end{itemize}
	Результат: Специалист ИЛИ администратор открыл страницу редактирования интересующего его отчёта.
\end{itemize}

\begin{itemize}[topsep=0pt, parsep=0pt, itemsep=0pt, leftmargin=*, labelindent=0.5cm]
	\item \textit{Фильтрация отчётов по статусу или типу} \\
	Действующее лицо: специалист ИЛИ администратор. \\
	Основной сценарий:
	\begin{itemize}[topsep=0pt, parsep=0pt, itemsep=0pt, leftmargin=*, labelindent=0.5cm]
		\item Специалист ИЛИ администратор открывает вкладку отчётов.
		\item В меню справа специалист ИЛИ администратор настраивает фильтрацию отчётов по статусу или типу.
		\item Отображается интересующая специалиста ИЛИ администратора выборка отчётов.
	\end{itemize}
	Результат: Специалист ИЛИ администратор отфильтровал отчёты в соответствии с интересующими его признаками.
\end{itemize}

\begin{itemize}[topsep=0pt, parsep=0pt, itemsep=0pt, leftmargin=*, labelindent=0.5cm]
	\item \textit{Поиск отчётов по дате отправки} \\
	Действующее лицо: специалист ИЛИ администратор. \\
	Основной сценарий:
	\begin{itemize}[topsep=0pt, parsep=0pt, itemsep=0pt, leftmargin=*, labelindent=0.5cm]
		\item Специалист ИЛИ администратор открывает вкладку отчётов.
		\item В меню сверху специалист ИЛИ администратор выбирает интересующую его дату отправки.
		\item Отображается интересующая специалиста ИЛИ администратора выборка отчётов.
	\end{itemize}
	Результат: Специалист ИЛИ администратор нашёл отчёты, отправленные за интересующую его дату.
\end{itemize}

\begin{itemize}[topsep=0pt, parsep=0pt, itemsep=0pt, leftmargin=*, labelindent=0.5cm]
	\item \textit{Сортировка отчётов} \\
	Действующее лицо: специалист ИЛИ администратор. \\
	Основной сценарий:
	\begin{itemize}[topsep=0pt, parsep=0pt, itemsep=0pt, leftmargin=*, labelindent=0.5cm]
		\item Специалист ИЛИ администратор открывает вкладку отчётов.
		\item В меню сверху специалист ИЛИ администратор выбирает поле отчёта для сортировки.
		\item Специалист ИЛИ администратор щёлкает по нему для сортировки ИЛИ щёлкает ещё раз для изменения порядка сортировки.
		\item Отображаются отчёты в отсортированном виде.
	\end{itemize}
	Результат: Специалист ИЛИ администратор отсортировал отчёты по интересующему его признаку.
\end{itemize}

\begin{itemize}[topsep=0pt, parsep=0pt, itemsep=0pt, leftmargin=*, labelindent=0.5cm]
	\item \textit{Выполнение действия с отчётом} \\
	Действующее лицо: специалист ИЛИ администратор. \\
	Основной сценарий:
	\begin{itemize}[topsep=0pt, parsep=0pt, itemsep=0pt, leftmargin=*, labelindent=0.5cm]
		\item Специалист ИЛИ администратор открывает вкладку отчётов.
		\item В меню сверху специалист ИЛИ администратор выбирает действие.
		\item Специалист ИЛИ администратор выставляет флажки справа от тех отчётов, к которым он хотел бы применить это действие.
		\item Специалист ИЛИ администратор нажимает на кнопку \textquote{\en{go}}.
		\item Выбранное действие производится с выбранными отчётами.
	\end{itemize}
	Результат: Специалист ИЛИ администратор применил выбранное действие к выбранным отчётам.
\end{itemize}

\begin{itemize}[topsep=0pt, parsep=0pt, itemsep=0pt, leftmargin=*, labelindent=0.5cm]
	\item \textit{Редактирование отчёта} \\
	Действующее лицо: специалист ИЛИ администратор. \\
	Основной сценарий:
	\begin{itemize}[topsep=0pt, parsep=0pt, itemsep=0pt, leftmargin=*, labelindent=0.5cm]
		\item Специалист ИЛИ администратор открывает вкладку отчётов.
		\item Специалист ИЛИ администратор выбирает интересующий его отчёт.
		\item Специалист ИЛИ администратор нажимает на его дату отправки.
		\item Открывается окно редактирования отчёта.
		\item Специалист ИЛИ администратор редактирует информацию об отчёте.
		\item Специалист ИЛИ администратор нажимает кнопку \textquote{\en{Save}} или \textquote{\en{Save and continue editing}}.
		\item В зависимости от нажатой кнопки, открывается вкладка отчётов или окно редактирования отчёта.
	\end{itemize}
	Результат: Специалист ИЛИ администратор отредактировал информацию выбранного отчёта.
\end{itemize}

\begin{itemize}[topsep=0pt, parsep=0pt, itemsep=0pt, leftmargin=*, labelindent=0.5cm]
	\item \textit{Удаление отчёта} \\
	Действующее лицо: специалист ИЛИ администратор. \\
	Основной сценарий:
	\begin{itemize}[topsep=0pt, parsep=0pt, itemsep=0pt, leftmargin=*, labelindent=0.5cm]
		\item Специалист ИЛИ администратор открывает вкладку отчётов.
		\item Специалист ИЛИ администратор выбирает интересующий его отчёт.
		\item Специалист ИЛИ администратор нажимает на его дату отправки.
		\item Открывается окно редактирования отчёта.
		\item Специалист ИЛИ администратор нажимает кнопку \textquote{\en{Delete}}.
		\item Открывается вкладка отчётов.
	\end{itemize}
	Результат: Специалист ИЛИ администратор удалил выбранный отчёт.
\end{itemize}

\begin{itemize}[topsep=0pt, parsep=0pt, itemsep=0pt, leftmargin=*, labelindent=0.5cm]
	\item \textit{Поиск профиля пользователя по роли} \\
	Действующее лицо: администратор. \\
	Основной сценарий:
	\begin{itemize}[topsep=0pt, parsep=0pt, itemsep=0pt, leftmargin=*, labelindent=0.5cm]
		\item Администратор открывает вкладку профилей пользователей.
		\item В меню справа администратор настраивает фильтрацию профилей пользователей по роли.
		\item Отображается интересующая администратора выборка профилей пользователей.
	\end{itemize}
	Результат: Администратор отфильтровал профили пользователей системы по роли.
\end{itemize}

\begin{itemize}[topsep=0pt, parsep=0pt, itemsep=0pt, leftmargin=*, labelindent=0.5cm]
	\item \textit{Сортировка профилей пользователей} \\
	Действующее лицо: администратор. \\
	Основной сценарий:
	\begin{itemize}[topsep=0pt, parsep=0pt, itemsep=0pt, leftmargin=*, labelindent=0.5cm]
		\item Администратор открывает вкладку профилей пользователей.
		\item В меню сверху администратор выбирает поле профиля пользователя для сортировки.
		\item Администратор щёлкает по нему для сортировки ИЛИ щёлкает ещё раз для изменения порядка сортировки.
		\item Отображаются профили пользователей в отсортированном виде.
	\end{itemize}
	Результат: Администратор отсортировал профили пользователей по интересующему его признаку.
\end{itemize}

\begin{itemize}[topsep=0pt, parsep=0pt, itemsep=0pt, leftmargin=*, labelindent=0.5cm]
	\item \textit{Выполнение действия с профилем пользователя} \\
	Действующее лицо: администратор. \\
	Основной сценарий:
	\begin{itemize}[topsep=0pt, parsep=0pt, itemsep=0pt, leftmargin=*, labelindent=0.5cm]
		\item Администратор открывает вкладку профилей пользователей.
		\item В меню сверху администратор выбирает действие.
		\item Администратор выставляет флажки справа от тех профилей пользователей, к которым он хотел бы применить это действие.
		\item Администратор нажимает на кнопку \textquote{\en{go}}.
		\item Выбранное действие производится с выбранными профилями пользователей.
	\end{itemize}
	Результат: Администратор применил выбранное действие к выбранным профилям пользователей.
\end{itemize}

\begin{itemize}[topsep=0pt, parsep=0pt, itemsep=0pt, leftmargin=*, labelindent=0.5cm]
	\item \textit{Редактирование профиля пользователя} \\
	Действующее лицо: администратор. \\
	Основной сценарий:
	\begin{itemize}[topsep=0pt, parsep=0pt, itemsep=0pt, leftmargin=*, labelindent=0.5cm]
		\item Администратор открывает вкладку профилей пользователей.
		\item Администратор выбирает интересующий его профиль пользователя.
		\item Администратор нажимает на его адрес электронной почты.
		\item Открывается окно редактирования профиля пользователя.
		\item Администратор редактирует информацию о профиле пользователя.
		\item Администратор нажимает кнопку \textquote{\en{Save}} или \textquote{\en{Save and continue editing}}.
		\item В зависимости от нажатой кнопки, открывается вкладка профилей пользователей или окно редактирования профиля пользователя.
	\end{itemize}
	Результат: Администратор отредактировал информацию выбранного профиля пользователя.
\end{itemize}

\begin{itemize}[topsep=0pt, parsep=0pt, itemsep=0pt, leftmargin=*, labelindent=0.5cm]
	\item \textit{Блокирование профиля пользователя} \\
	Действующее лицо: администратор. \\
	Основной сценарий:
	\begin{itemize}[topsep=0pt, parsep=0pt, itemsep=0pt, leftmargin=*, labelindent=0.5cm]
		\item Администратор открывает вкладку профилей пользователей.
		\item Администратор выбирает интересующий его профиль пользователя.
		\item Администратор нажимает на его адрес электронной почты.
		\item Открывается окно редактирования профиля пользователя.
		\item Администратор снимает флажок поля \textquote{\en{Active}}.
		\item Администратор нажимает кнопку \textquote{\en{Save}} или \textquote{\en{Save and continue editing}}.
		\item В зависимости от нажатой кнопки, открывается вкладка профилей пользователей или окно редактирования профиля пользователя.
	\end{itemize}
	Результат: Администратор заблокировал выбранный профиль пользователя, данный пользователь, специалист или администратор больше не сможет войти в систему.
\end{itemize}

\begin{itemize}[topsep=0pt, parsep=0pt, itemsep=0pt, leftmargin=*, labelindent=0.5cm]
	\item \textit{Удаление отчёта} \\
	Действующее лицо: администратор. \\
	Основной сценарий:
	\begin{itemize}[topsep=0pt, parsep=0pt, itemsep=0pt, leftmargin=*, labelindent=0.5cm]
		\item Администратор открывает вкладку профилей пользователей.
		\item Администратор выбирает интересующий его профиль пользователя.
		\item Администратор нажимает на его адрес электронной почты.
		\item Открывается окно редактирования профиля пользователя.
		\item Администратор нажимает кнопку \textquote{\en{Delete}}.
		\item Открывается вкладка профилей пользователей.
	\end{itemize}
	Результат: Администратор удалил выбранный профиль пользователя, все связанные с ним отчёты сохранились.
\end{itemize}

\begin{itemize}[topsep=0pt, parsep=0pt, itemsep=0pt, leftmargin=*, labelindent=0.5cm]
	\item \textit{Выход с сервисного сайта} \\
	Действующее лицо: специалист ИЛИ администратор. \\
	Основной сценарий:
	\begin{itemize}[topsep=0pt, parsep=0pt, itemsep=0pt, leftmargin=*, labelindent=0.5cm]
		\item Специалист ИЛИ администратор наживает на ссылку \textquote{Выйти} в правом верхнем углу экрана.
		\item Отображается страница входа в систему.
	\end{itemize}
	Результат: Специалист вышел из своей учётной записи и теперь в систему с того же устройства может войти другой пользователь.
\end{itemize}

\subsection{Политика конфиденциальности}

\tab
Так как для регистрации в системе пользователям необходимо указать и подтвердить их адрес электронной почты, а также в процессе работы с системой они могут загружать в неё фотоматериалы, администрации системы необходимо получить информированное согласие пользователей на обработку их конфиденциальных данных\cite{privacy-policy}.

\tab
Для проектируемой системы был составлен пример политики конфиденциальности (детали могут различаться в зависимости от целей и задач организации, использующей систему), её основные положения должны заключаться в следующем:
\begin{itemize}
	\item Система в обязательном порядке собирает и хранит личные данные пользователей, необходимые для связи с ними в экстренной ситуации (их адрес электронной почты).
	\item Администрация системы имеет право использовать эти данные для связи с пользователями.
	\item Эти данные не передаются третьим лицам (в том числе и другим пользователям).
	\item Такие данные навсегда удаляются из системы вместе с аккаунтом пользователя.
	\item Все остальные личные данные (такие как имя пользователя и фотографии, прикреплённые к оставляемым им отчётам) оставляются пользователем исключительно по его желанию.
	\item Такие данные могут обрабатываться автоматически или вручную неограниченным кругом лиц и будут распространяться открыто.
	\item Такие данные не удаляются из системы при удалении аккаунта пользователя, так как могут представлять статистическую или исследовательскую ценность, но могут быть удалены по запросу пользователя или в связи с решением администрации системы.
\end{itemize}

\tab
Пример политики конфиденциальности представлен в \additionref{privacy-policy}.

\subsection{Модель данных}

\tab
База данных была спроектирована с помощью ER-метода (метода \textquote{сущность‐связь}).
Для этого была создана ER‐модель, показывающая как атрибуты сущностей предметной области, так и их связи.
Атрибут или атрибуты, уникально идентифицирующие сущность, называются ключевыми, далее они подчёркнуты.

\subsubsection{Описание сущностей}

\tab
В предметной области можно выделить следующие сущности:
\begin{itemize}
	\item \textquote{Пользователь} с ключом \textquote{Номер пользователя} и атрибутами: \textquote{Адрес электронной почты}, \textquote{Хэш сумма пароля}, \textquote{Имя}, \textquote{Дата регистрации в системе}, \textquote{Время последнего входа в систему}, \textquote{Является ли пользователь администратором}, \textquote{Является ли пользователь специалистом}, \textquote{Активен ли аккаунт пользователя}, \textquote{Фото профиля} и \textquote{Уменьшенное фото профиля}. Адрес электронной почты также уникален для сущности \textquote{Пользователь}, но, согласно политике конфиденциальности, его нельзя распространять при помощи публичного API, поэтому использовать его в качестве ключа невозможно.
	\item \textquote{Отчёт} с ключом \textquote{Номер отчёта} и атрибутами: \textquote{Дата и время наблюдения}, \textquote{Адрес}, \textquote{Первый комментарий}, \textquote{Координаты}, \textquote{Статус} и \textquote{Вид растения}.
	\item \textquote{Фотография} с ключом \textquote{Номер фотографии} и атрибутами: \textquote{Фотография} и \textquote{Уменьшенная фотография}
	\item \textquote{Комментарий} с ключом \textquote{Номер комментария} и с атрибутом: \textquote{Текст комментария}.
\end{itemize}

\subsubsection{ER модель}

\tab
Между сущностями можно установить следующие связи:
\begin{itemize}
	\item \textquote{Пользователь - Отчёт}: пользователь оставляет отчёт о наблюдении потенциально опасного для человека растения. Пользователь может не оставлять отчётов, оставить один или несколько. Отчёт может быть оставлен существующим или удалённым пользователем. Связь не обязательная с обеих сторон, имеет степень 1 со стороны пользователя и n со стороны отчёта.
	\item \textquote{Пользователь - Комментарий}: пользователь оставляет комментарий к отчёту. Пользователь может не оставлять комментариев, оставить один или несколько. Комментарий может быть оставлен существующим или удалённым пользователем. Связь не обязательная с обеих сторон, имеет степень 1 со стороны пользователя и n со стороны комментария.
	\item \textquote{Отчёт - Комментарий}: комментарий относится к какому-либо отчёту. К отчёту может не относиться ни одного комментария, относиться один или несколько. Комментарий относится только к одному отчёту. Связь обязательная со стороны комментария и не обязательная со стороны отчёта, имеет степень 1 со стороны отчёта и n со стороны комментария.
	\item \textquote{Отчёт - Фотография}: фотография прикреплена к какому-либо отчёту. К каждому отчёту может быть не прикреплено ни одной фотографии, прикреплена одна или несколько. Каждая фотография прикреплена к какому-либо отчёту. Связь обязательная со стороны фотографии и не обязательная со стороны отчёта, имеет степень 1 со стороны отчёта и n со стороны комментария.
\end{itemize}

\tab
На рисунке 2 представлена ER диаграмма модели.
\picturetwo

\subsubsection{Реляционная модель}

\tab
Связь \textquote{Пользователь - Отчёт} образует два отношения - по одному для каждой сущности, в отношение \textquote{Отчёт} добавляются ключевые атрибуты сущности \textquote{Пользователь}:
\begin{itemize}
	\item Пользователь (\under{Номер пользователя}).
	\item Отчёт (\under{Номер отчёта}, Номер пользователя).
\end{itemize}

\tab
Связь \textquote{Пользователь - Комментарий} образует два отношения - по одному для каждой сущности, в отношение \textquote{Комментарий} добавляются ключевые атрибуты сущности \textquote{Пользователь}:
\begin{itemize}
	\item Пользователь (\under{Номер пользователя}).
	\item Комментарий (Номер отчёта, Номер пользователя).
\end{itemize}

\tab
Связь \textquote{Отчёт - Комментарий} образует два отношения - по одному для каждой сущности, в отношение \textquote{Комментарий} добавляются ключевые атрибуты сущности \textquote{Отчёт}:
\begin{itemize}
	\item Отчёт (\under{Номер отчёта}, Номер пользователя).
	\item Комментарий (Номер отчёта).
\end{itemize}

\tab
Связь \textquote{Отчёт - Фотография} образует два отношения - по одному для каждой сущности, в отношение \textquote{Фотография} добавляются ключевые атрибуты сущности \textquote{Отчёт}:
\begin{itemize}
	\item Отчёт (\under{Номер отчёта}, Номер пользователя).
	\item Фотография (Номер отчёта).
\end{itemize}

\tab
Оставшиеся атрибуты предметной области были распределены между отношениями следующим образом:
\begin{itemize}
	\item Пользователь (\under{Номер пользователя}, Адрес электронной почты, Хэш сумма пароля, Имя, Дата регистрации в системе, Время последнего входа в систему, Является ли пользователь администратором, Является ли пользователь специалистом, Активен ли аккаунт пользователя, Фото профиля, Уменьшенное фото профиля).
	\item Отчёт (\under{Номер отчёта}, Номер пользователя, Дата и время наблюдения, Адрес, Первый комментарий, Координаты, Статус, Вид растения).
	\item Фотография (\under{Номер фотографии}, Номер отчёта, Фотография, Уменьшенная фотография).
	\item Комментарий (\under{Номер комментария}, Номер пользователя, Номер отчёта, Текст комментария).
\end{itemize}

\tab
Все отношения отражают объекты или зависимости предметной области, избыточных отношений нет.
На рисунке 3 представлена реляционная модель базы данных.
\picturethree

\subsection{Описание пользовательского интерфейса}

\tab
В соответствии со сценариями использования был разработан макет пользовательского интерфейса. Рисунки макета находятся в \additionref{interface-plan}.

\subsubsection{Пользовательский интерфейс клиентского приложения}

\tab
При запуске клиентского приложения открывается его главный экран (\subadditionref{main-screen-plan}).
На нём изображена карта с центром в Санкт-Петербурге.
На карте иконками растений отмечены места, в которых пользователи системы отметили нахождение отслеживаемых растений.
Цвет иконок на карте различается в зависимости от статуса отчёта (\textquote{необработано} - зелёный, \textquote{актуально} - оранжевый и \textquote{неактуально} - серый).
Если пользователь дал приложению доступ к определению своего местоположения, на карте также будет отмечена его позиция иконкой в виде человека, а сама карта изначально будет центрирована по его местоположению.
\begin{enumerate}
	\item При нажатии на кнопку навигации в правом верхнем углу экрана открывается панель навигации (\subadditionref{main-screen-plan}).
	Вверху панели отображается фото профиля пользователя и его имя в случае, если он установил себе имя, или же адрес электронной почты.
	\begin{enumerate}
		\item При нажатии на кнопку \textquote{Account} открывается экран аккаунта пользователя (\subadditionref{account-screen-plan}).
		Вверху экрана отображается фото профиля пользователя, которое можно заменить при нажатии на него.
		Также пользователю предлагается изменить его имя, адрес электронной почты, пароль, а также выйти из системы и удалить аккаунт.
		\begin{enumerate}
			\item При изменении электронной почты после ввода нового адреса открывается диалоговое окно, в которое для подтверждения изменения пользователь должен ввести код из полученного им письма (\subadditionref{code-dialog-plan}).
			\item При изменении пароля открывается диалоговое окно, в которое для подтверждения изменения пользователь должен ввести новый пароль дважды (\subadditionref{password-dialog-plan}).
		\end{enumerate}
		В случае, если пользователь ещё не произвёл авторизацию в системе, открывается экран авторизации (\subadditionref{auth-screen-plan}).
		На этом экране пользователю предлагается ввести адрес электронной почты создаваемого аккаунта и дважды повторить его пароль.
		Также для создания профиля пользователю необходимо поставить флажок о том, что он согласен с политикой конфиденциальности данных.
		\begin{enumerate}
			\item При нажатии на кнопку \textquote{\en{Already have account?}} открывается экран входа в систему (\subadditionref{login-screen-plan}).
			От экрана авторизации он отличается тем, что для входа систему нет необходимости повторно подтверждать пароль, электронную почту и соглашаться с политикой конфиденциальности данных.
			Обратно на экран авторизации с него можно попасть, нажав на кнопку \textquote{\en{New to HogWeedGo?}}.
			\item При нажатии на кнопку \textquote{\en{Already have account?}} открывается диалоговое окно, в которое для подтверждения адреса электронной почты пользователь должен ввести код из полученного им письма (\subadditionref{code-dialog-plan}).
			\item При нажатии на слова \textquote{\en{Terms and conditions}}, помеченные голубым в описании флажка согласия с политикой конфиденциальности, пользователь перенаправляется на страницу с опубликованной политикой конфиденциальности системы и управляющей ей системы.
		\end{enumerate}
		\item При нажатии на кнопку \textquote{\en{Map}} открывается главный экран приложения с картой.
		\item При нажатии на кнопку \textquote{\en{About}} открывается экран с информацией о системе и об организации, которая ей управляет.
		\item При нажатии на кнопку \textquote{\en{Feedback}} пользователь перенаправляется на страницу ссвязи с организацией, управляющей системой.
	\end{enumerate}
	\item При нажатии на отметку на карте открывается панель демонстрации отчёта (\subadditionref{view-screen-plan}).
	На этой панели пользователю демонстрируется информация данного отчёта, фотографии, приложенные к нему, а также комментарии, дополняющие отчёт.
	Пользователю предлагается также оставить комментарий к отчёту.
	\begin{enumerate}
		\item При нажатии на приложенную к отчёту фотографию, она открывается в полноэкранном режиме (\subadditionref{full-screen-plan}).
		При нажатии на кнопку с символом дискеты в левом нижнем углу экрана пользователю предлагается скачать эту фотографию.
	\end{enumerate}
	\item При нажатии на кнопку \textquote{+} в левом нижнем углу экрана открывается экран отправки отчёта(\subadditionref{report-screen-plan}).
	На этом экране пользователю предлагается ввести данные, необходимые для отправки отчёта (краткое описание найденного растения, его адрес, место находки на карте, а также дату и время находки).
	По нажатию на кнопку \textquote{\en{Send}} отчёт отправляется в систему.
\end{enumerate}

\subsubsection{Пользовательский интерфейс сервисного сайта}

Макетом для сервисного сайта системы послужил автоматически сгенерированный фреймворком \en{Django} сайт.
При запуске сервисного сайта открывается страница входа (\subadditionref{login-page-plan}).
После авторизации пользователя в системе открывается главная страница сайта (\subadditionref{main-page-plan}).
На ней справа отображаются последние действия, совершенные в системе, снизу на карте отмечены места, в которых пользователи системы отметили нахождение отслеживаемых растений.
При нажатии на такую отметку открывается страница редактирования соответствующего отчёта (\subadditionref{report-page-plan}).
\begin{enumerate}
	\item При нажатии на кнопку \textquote{\en{Users}} открывается страница отображения пользователей системы (\subadditionref{users-page-plan}).
	На этой странице с профилями пользователей можно проводить массовые действия (см п. \ref{subsec:serverside-app}), искать их, сортировать и фильтровать.
	\begin{enumerate}
		\item При нажатии на адрес электронной почты пользователя открывается страница редактирования соответствующего профиля пользователя (\subadditionref{user-page-plan}).
		На этой странице профиль можно редактировать, блокировать, удалять, а также написать пользователю электронное письмо, перейдя по ссылке в соответствующем поле.
	\end{enumerate}
	\item При нажатии на кнопку \textquote{\en{Reports}} открывается страница отображения отчётов (\subadditionref{reports-page-plan}).
	На этой странице с отчётами можно проводить массовые действия (см п. \ref{subsec:serverside-app}), искать их, сортировать и фильтровать.
	\begin{enumerate}
		\item При нажатии на дату отправки открывается страница редактирования соответствующего отчёта (\subadditionref{report-page-plan}).
		На этой странице отчёт можно редактировать, удалять, а также редактировать и удалять приложенные к нему комментарии и фотографии.
		При нажатии на ссылку в соответствующем поле открывается редактирования профиля пользователя, отправившего отчёт.
	\end{enumerate}
\end{enumerate}

\tab
Рисунки экрана приложений, разработанных по макету, находятся в \additionref{interface-real} и имеют соответствующую нумерацию.
