% !TeX spellcheck = russian-aot

\newcont*{Список Литературы}

\bibliographystyle{unsrtnat}

\begin{thebibliography}{99}
	\bibitem{alien-plants}
	https://www.ncbi.nlm.nih.gov/pmc/articles/PMC7194640/
	\bibitem{invasive-plants}
	https://www.sciencedaily.com/releases/2022/02/220210125820.htm
	\bibitem{heracleum} https://cyberleninka.ru/article/n/borschevik-sosnovskogo-v-rossiyskoy-federatsii/viewer
	\webbibitem{rivr}{https://ib.komisc.ru/add/rivr}{Официальный сайт проекта РИВР}{26.12.2021}
	\webbibitem{esquire}{https://esquire.ru/articles/5849-cow-parsnip}{Статья в журнале Esquire, посвященная борщевику Cосновского}{12.01.2022}
	
	\webbibitem{googlemymaps}{https://www.inaturalist.org}{Официальный сайт проекта iNaturalist}{26.12.2021}
	\webbibitem{inaturalist}{https://www.google.com/maps/about/mymaps}{Страница с описанием сервиса Google MyMaps}{26.12.2021}
	\webbibitem{plantfinder}{https://www.plantsmap.com/finder/plants}{Страница с описанием сервиса Plant Finder}{26.12.2021}
	\webbibitem{floraincognita}{https://floraincognita.com}{Официальный сайт сервиса Flora Incognita}{11.01.2022}
	
	
	
	\bibitem{postgis-vs-mongo} Bartoszewski, Dominik \& Piórkowski, Adam \& Lupa, Michał. (2019). The Comparison of Processing Efficiency of Spatial Data for PostGIS and MongoDB Databases. 10.1007/978-3-030-19093-4\_22.
	\webbibitem{flutter-vs-react}{https://dergipark.org.tr/en/download/article-file/1607441}{Сравнение
	популярных фреймворков для кроссплатформенной разработки}{18.05.2022}
	\webbibitem{flutter-better}{http://www.diva-portal.org/smash/get/diva2:1442804/FULLTEXT01.pdf}{Оценка производительности популярных фреймворков для кроссплатформенной разработки}{18.05.2022}
	\webbibitem{bloc-better}{https://flutterdevs.com/blog/bloc-pattern-in-flutter-part-1}{Описание архитектуры BLoC}{18.05.2022}
	\bibitem{convolutional-better} Shamsaldin, Ahmed \& Fattah, Polla \& Rashid, Tarik \& Al-Salihi, Nawzad. (2019). The Study of The Convolutional Neural Networks Applications. UKH Journal of Science and Engineering. 3. 31-40. 10.25079/ukhjse.v3n2y2019.pp31-40.
	\bibitem{tensorflow-better} Kiran, T. Tritva Jyothi, Computer Vision Accuracy Analysis with Deep Learning Model Using TensorFlow (JULY 22, 2020). International Journal of Innovative Research in Computer Science \& Technology (IJIRCST) ISSN: 2347-5552, Volume, 8, Issue, 4, July, 2020, Available at SSRN: \href{https://ssrn.com/abstract=3673214}{https://ssrn.com/abstract=3673214} or \href{http://dx.doi.org/10.2139/ssrn.3673214}{http://dx.doi.org/10.2139/ssrn.3673214} 
	\webbibitem{xception-better}{https://arxiv.org/abs/1610.02357}{Описание модели XCeption свёрточной нейронной сети, обученной на наборе данных ImageNet}{18.05.2022}
	
	
	
	\webbibitem{un-biodiv}{https://www.cbd.int/convention/text/}{United Nations Convention on Biological Diversity (CBD)}{17.05.2022}
	\bibitem{poisonous-ivy}
	nytimes.com/interactive/projects/cp/climate/2015-paris-climate-talks/what-climate-change-looks-like-poison-ivy
	\bibitem{heracleum-desc} Виноградова В.М. Борщевик — Heracleum L. // Флора Восточной Европы. — М.; СПб.: Товарищество научных изданий КМК, 2004, Т. 11. – С. 404.
	\bibitem{kommersant} https://www.kommersant.ru/doc/2781947
	\bibitem{er-initiative} https://er.ru/activity/news/zachem-neobhodima-federalnaya-programma-po-borbe-s-borshevikom-razyasnyaet-vladimir-burmatov
	\bibitem{mos-expenses} https://www.agroinvestor.ru/business-pages/34172-sovremennye-metodiki-sposobny-sokratit-raskhody-na-borbu-s-borshchevikom-v-neskolko-raz/
	\bibitem{len-expenses} https://www.lenoblzaks.ru/publics/single/115/63977
	\bibitem{rural-measures} https://fsvps.gov.ru/ru/fsvps/news/36717.html
	\bibitem{countryside-measures} "Земельный кодекс Российской Федерации" от 25.10.2001 N 136-ФЗ (ред. от 01.05.2022)
	\bibitem{rural-law} 
	Федеральный закон "Об обороте земель сельскохозяйственного назначения" от 24.07.2002 N 101-ФЗ
	\webbibitem{business-online}{https://www.business-gazeta.ru/article/412341}{Пересказ заседания Государственного Совета, опубликованный в газете Бизнес Online}{17.05.2022}
	\bibitem{govern-law} Распоряжение от 25 ноября 2019 г. N 35-р об утверждении методических рекомендаций по структуре и содержанию программ сохранения биологического разнообразия коммерческих организаций	
	\bibitem{nature-law} Распоряжение Минприроды России от 25.11.2019 N 35-р "Об утверждении Методических рекомендаций по структуре и содержанию программ сохранения биологического разнообразия коммерческих организаций"
\end{thebibliography}
