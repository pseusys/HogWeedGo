% !TeX spellcheck = en_US

\makeatletter
\def\trueonehalfspace{1.213}

\newcommand{\titlelongtext}[1]{\begin{spacing}{\trueonehalfspace}#1\end{spacing}}

\newcommand{\@subscript}[1]{
	\fontsize{12}{14.4}\centering{\textit{\textsuperscript{#1}}}
}

\newcommand{\emptydate}{
	\textquote{\uline{\hspace{1cm}}} \uline{\hspace{3cm}} 2022 г.
}

\newcommand{\worktitle}[1]{
	\begin{tabularx}{\textwidth}{ X X }
		Студент \hspace{1cm} Сергеев А. Д. & \raggedleft{Группа 8304}
	\end{tabularx}
	\begin{tabularx}{\textwidth}{ X }
		\titlelongtext{
			Тема работы: Разработка программной системы наблюдения за распространением потенциально  опасных для человека видов растений
		}
		#1
	\end{tabularx}
}

\newcommand{\subsblock}{
	\begin{tabularx}{\textwidth}[t]{ X X X X }
		Студент & & & Сергеев А. Д. \\ \cline{3-3}
		& & \@subscript{(подпись)} & \\
		Руководитель & \centering{к.т.н., доцент} & & Заславский М. М. \\ \cline{3-3}
		& \@subscript{(Уч. степень, уч. звание)} & \@subscript{(подпись)} & \\
		Консультант & \centering{асс. каф. ИЗОС} & & Контрош Л. В. \\ \cline{3-3}
		& \@subscript{(Уч. степень, уч. звание)} & \@subscript{(подпись)} & \\
		Консультант & \centering{к.т.н., доцент} & & Заславский М. М. \\ \cline{3-3}
		& \@subscript{(Уч. степень, уч. звание)} & \@subscript{(подпись)} & \\
		Консультант & & & Черепанов И. В. \\ \cline{3-3}
		& & \@subscript{(подпись)} & \\
	\end{tabularx}
}

\newcommand{\confirmation}{
	\setlength{\extrarowheight}{0.25cm}
	\begin{tabularx}{\textwidth}{ X r }
		& Утверждаю \\
		& Зав. кафедрой МО ЭВМ \\
		& \uline{\hspace{3cm}} Кринкин К. В. \\
		& \emptydate
	\end{tabularx}
}



\let\latex@input\input
\newcommand\current@input[1]{\def\currentfile{#1} \par\nobreak\latex@input{#1}%
}
\AtBeginDocument{\let\input\current@input}



\newcommand{\defineterm}[2]{\label{def:#1}\hyperref[lnk:#1]{#2}}
\newcommand{\linkterm}[2]{\label{lnk:#1}\hyperref[def:#1]{\textbf{#2}}}

\newcommand{\tab}{\hspace{0.5cm}}
\newcommand{\nwln}{\vspace{\baselineskip}\tab}

\newcommand{\en}[1]{\foreignlanguage{english}{#1}}
\newcommand{\de}[1]{\foreignlanguage{german}{#1}}
\newcommand{\la}[1]{\foreignlanguage{latin}{#1}}

\newcommand{\longcaption}[1]{
	\caption{#1}
	\addtocounter{table}{-1}
	\endfirsthead
	\caption*{Продолжение таблицы \arabic{table}}
	\endhead
}



\titleformat{\section}[hang] {\centering\bfseries}{\thesection\ } {0pt}{\MakeUppercase}
\titleformat{\subsection}[hang] {\bfseries}{\thesubsection\ } {0pt}{}
\renewcommand{\thesection}{\arabic{section}}

\newcommand{\@newcont}[1]{\addcontentsline{toc}{section}{#1}}
\newcommand{\@newcontsect}[1]{\phantomsection\section*{#1}\@newcont{#1}}
\newcommand{\@newcontnosect}[1]{\phantomsection\@newcont{#1}}
\newcommand{\newcont}{\@ifstar{\@newcontnosect}{\@newcontsect}}



\addto\captionsrussian{\renewcommand{\figurename}{Рисунок}}

\tikzset{node distance = 3.5cm and 5.5cm}
\tikzset{block/.style = {draw, minimum height=1.5cm, text width=3cm, align=center}}
\tikzset{block/.style = {draw, minimum height=1.5cm, text width=3cm, align=center}}
\tikzset{link/.style = {minimum height=1cm, text width=2.25cm, midway, align=center}}



\titleformat{\chapter}[display] {\centering\bfseries}{} {0pt}{\MakeUppercase}
\titlespacing*{\chapter}{0pt}{-\baselineskip}{0pt}

\let\@thebibliography\thebibliography
\renewcommand\thebibliography[1]{
	\@thebibliography{#1}
	\setlength{\parskip}{0pt}
	\setlength{\itemsep}{0pt plus 0.3ex}
}

\renewcommand\bibname{Список литературы}
\newcommand{\webbibitem}[4]{\bibitem{#1} \href{#2}{#2} // #3, дата обращения: #4}
\makeatother
