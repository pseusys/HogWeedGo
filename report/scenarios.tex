% !TeX spellcheck = russian-aot

\subsubsection{Сценарии использования клиентского приложения}

\tab
Действующее лицо: пользователь.

\begin{itemize}[topsep=0pt, parsep=0pt, itemsep=0pt, leftmargin=*, labelindent=0.5cm]
	\item \textit{Просмотр опубликованной отсортированной информации о наблюдениях} \\
	Основной сценарий:
	\begin{itemize}[topsep=0pt, parsep=0pt, itemsep=0pt, leftmargin=*, labelindent=0.5cm]
		\item Пользователь открывает клиентское приложение.
		\item Пользователь нажимает на один из указателей на карте, ему демонстрируются подробные сведения об отчёте и комментарии к нему.
		\item По желанию пользователь нажимает на фотографии, прикреплённые к отчёту и просматривает их увеличенную версию.
	\end{itemize}
	Результат: Пользователь ознакомился как с общей статистикой распространения потенциально опасных растений на карте, так и с заинтересовавшими его конкретными отчётами.
	
	\item \textit{Просмотр информации о системе, её задачах и краткого руководства по использованию} \\
	Основной сценарий:
	\begin{itemize}[topsep=0pt, parsep=0pt, itemsep=0pt, leftmargin=*, labelindent=0.5cm]
		\item На основной странице приложения пользователь открывает меню навигации и выбирает пункт \textquote{\en{About}}, ему демонстрируется информация о системе, её задачах и краткое руководство по использованию.
	\end{itemize}
	Результат: Пользователь посмотрел данные со всей информацией по операциям конкретного изделия.
	
	\item \textit{Авторизация в системе} \\
	Условие: На основной странице приложения пользователь открывает меню навигации и выбирает пункт \textquote{\en{Account}} ИЛИ на основной странице приложения нажимает на кнопку справа внизу с символом \textquote{+} ИЛИ на странице просмотра отчёта нажимает на ссылку \textquote{\en{Sign in}}. \\
	Основной сценарий:
	\begin{itemize}[topsep=0pt, parsep=0pt, itemsep=0pt, leftmargin=*, labelindent=0.5cm]
		\item Открывается страница авторизации пользователя.
		\item Пользователь нажимает кнопку \textquote{\en{Already have account?}}.
		\item Пользователь вводит свой адрес электронной почты в поле ввода адреса электронной почты.
		\item Пользователь вводит свой пароль в поле ввода пароля.
		\item Пользователь вводит свой пароль в поле ввода подтверждения пароля.
		\item Пользователь подтверждает согласие на сбор и обработку персональных данных, поставив флаг \textquote{\en{Agree to terms and conditions}}.
		\item Пользователь нажимает кнопку \textquote{\en{Submit}}.
		\item Пользователь подтверждает указанный адрес электронной почты, в открывшемся диалоговом окне он вводит восьмизначный код, отправленный на его электронную почту.
		\item Пользователь нажимает кнопку \textquote{\en{Submit}}.
		\item Диалоговое окно закрывается.
		\item В нижнем правом углу появляется сообщение о статусе авторизации и, в случае её успеха, открывается главная страница приложения.
	\end{itemize}
	Альтернативный сценарий (условие - наличие у пользователя аккаунта):
	\begin{itemize}[topsep=0pt, parsep=0pt, itemsep=0pt, leftmargin=*, labelindent=0.5cm]
		\item Открывается страница авторизации пользователя.
		\item Пользователь вводит свой адрес электронной почты в поле ввода адреса электронной почты.
		\item Пользователь вводит свой пароль в поле ввода пароля
		\item Пользователь нажимает кнопку \textquote{\en{Submit}}.
		\item В нижнем правом углу появляется сообщение о статусе авторизации и, в случае её успеха, открывается главная страница приложения.
	\end{itemize}
	Результат: Пользователь авторизован в системе и может выполнять требующие авторизации сценарии.
	
	\item \textit{Отправка отчёта о наблюдении потенциально опасного растения} \\
	Условие: Пользователь авторизован в системе. \\
	Основной сценарий:
	\begin{itemize}[topsep=0pt, parsep=0pt, itemsep=0pt, leftmargin=*, labelindent=0.5cm]
		\item На основной странице приложения нажимает на кнопку справа внизу с символом \textquote{+}.
		\item Снизу экрана открывается форма отправки отчёта.
		\item Пользователь заполняет поля описания находки и, по желанию, описания местоположения находки. При возможности поле описания местоположения находки заполняется автоматически (см. п. 3.5) текущим адресом пользователя.
		\item Пользователь выбирает местоположение своей находки на карте ИЛИ нажимает на кнопку \textquote{Использовать моё местонахождение}.
		\item Пользователь заполняет даты и времени обнаружения растения при помощи специальных виджетов выбора даты и времени.
		\item По желанию пользователь загружает фотографии находки. Фотографии проверяются системой автоматической проверки изображений и пользователь ознакамливается с результатами проверки справа от поля для загрузки фотографий.
		\item По желанию пользователь нажимает на фотографии, прикреплённые к отчёту и просматривает их увеличенную версию.
		\item По желанию пользователь оставляет общий комментарий к своей находке.
		\item Пользователь нажимает кнопку \textquote{\en{Send}}.
		\item Форма отправки отчёта закрывается.
		\item В нижнем правом углу появляется сообщение о статусе отправки и, в случае успеха, на карту добавляется новая метка в указанном месте.
	\end{itemize}
	Результат: Пользователь отправил отчёт о наблюдении потенциально опасного растения, его фотографии были предварительно проверены и данные загружены в систему.
	
	\item \textit{Написание комментария к отчёту} \\
	Условие: Пользователь авторизован в системе И на основной странице приложения пользователь нажимает на один из указателей на карте. \\
	Основной сценарий:
	\begin{itemize}[topsep=0pt, parsep=0pt, itemsep=0pt, leftmargin=*, labelindent=0.5cm]
		\item Пользователь вводит свой комментарий к отчёту в поле для ввода комментария и нажимает кнопку с символом \textquote{>}.
		\item Поле для ввода комментария очищается.
		\item В нижнем правом углу появляется сообщение о статусе отправки комментария.
	\end{itemize}
	Результат: Пользователь оставил комментарий, дополнив информацию, изложенную в отчёте.
	
	\item \textit{Редактирование профиля} \\
	Условие: Пользователь авторизован в системе. \\
	Основной сценарий:
	\begin{itemize}[topsep=0pt, parsep=0pt, itemsep=0pt, leftmargin=*, labelindent=0.5cm]
		\item На основной странице приложения пользователь открывает меню навигации и выбирает пункт \textquote{\en{Account}}.
		\item По желанию пользователь выбирает новую фотографию профиля.
		\item По желанию пользователь вводит новое имя и нажимает \textquote{\en{Rename}}.
		\item По желанию пользователь ...
		/\ /\ TODO: decide interface!!
		\item В нижнем правом углу появляется сообщение о статусе обновления профиля.
	\end{itemize}
	Результат: Пользователь оставил комментарий, дополнив информацию, изложенную в отчёте.
\end{itemize}

\subsubsection{Сценарии использования сервисного сайта}

\begin{itemize}[topsep=0pt, parsep=0pt, itemsep=0pt, leftmargin=*, labelindent=0.5cm]
	\item \textit{Вход на сервисный сайт} \\
	Действующее лицо: специалист ИЛИ администратор. \\
	Основной сценарий:
	\begin{itemize}[topsep=0pt, parsep=0pt, itemsep=0pt, leftmargin=*, labelindent=0.5cm]
		\item Специалист ИЛИ администратор вводит свой адрес электронной почты в поле ввода адреса электронной почты.
		\item Специалист ИЛИ администратор вводит свой пароль в поле ввода пароля.
	\end{itemize}
	Результат: Специалист ИЛИ администратор авторизован в системе и может выполнять требующие авторизации сценарии.
\end{itemize}

\begin{itemize}[topsep=0pt, parsep=0pt, itemsep=0pt, leftmargin=*, labelindent=0.5cm]
	\item \textit{Поиск отчета по карте} \\
	Действующее лицо: специалист ИЛИ администратор. \\
	Основной сценарий:
	\begin{itemize}[topsep=0pt, parsep=0pt, itemsep=0pt, leftmargin=*, labelindent=0.5cm]
		\item Специалист ИЛИ администратор находит интересующий его маркер отчёта на карте.
		\item Специалист ИЛИ администратор нажимает на него.
		\item Открывается страница редактирования отчёта.
	\end{itemize}
	Результат: Специалист ИЛИ администратор открыл страницу редактирования интересующего его отчёта.
\end{itemize}

\begin{itemize}[topsep=0pt, parsep=0pt, itemsep=0pt, leftmargin=*, labelindent=0.5cm]
	\item \textit{Фильтрация отчётов по статусу или типу} \\
	Действующее лицо: специалист ИЛИ администратор. \\
	Основной сценарий:
	\begin{itemize}[topsep=0pt, parsep=0pt, itemsep=0pt, leftmargin=*, labelindent=0.5cm]
		\item Специалист ИЛИ администратор открывает вкладку отчётов.
		\item В меню справа специалист ИЛИ администратор настраивает фильтрацию отчётов по статусу или типу.
		\item Отображается интересующая специалиста ИЛИ администратора выборка отчётов.
	\end{itemize}
	Результат: Специалист ИЛИ администратор отфильтровал отчёты в соответствии с интересующими его признаками.
\end{itemize}

\begin{itemize}[topsep=0pt, parsep=0pt, itemsep=0pt, leftmargin=*, labelindent=0.5cm]
	\item \textit{Поиск отчётов по дате отправки} \\
	Действующее лицо: специалист ИЛИ администратор. \\
	Основной сценарий:
	\begin{itemize}[topsep=0pt, parsep=0pt, itemsep=0pt, leftmargin=*, labelindent=0.5cm]
		\item Специалист ИЛИ администратор открывает вкладку отчётов.
		\item В меню сверху специалист ИЛИ администратор выбирает интересующую его дату отправки.
		\item Отображается интересующая специалиста ИЛИ администратора выборка отчётов.
	\end{itemize}
	Результат: Специалист ИЛИ администратор нашёл отчёты, отправленные за интересующую его дату.
\end{itemize}

\begin{itemize}[topsep=0pt, parsep=0pt, itemsep=0pt, leftmargin=*, labelindent=0.5cm]
	\item \textit{Сортировка отчётов} \\
	Действующее лицо: специалист ИЛИ администратор. \\
	Основной сценарий:
	\begin{itemize}[topsep=0pt, parsep=0pt, itemsep=0pt, leftmargin=*, labelindent=0.5cm]
		\item Специалист ИЛИ администратор открывает вкладку отчётов.
		\item В меню сверху специалист ИЛИ администратор выбирает поле отчёта для сортировки.
		\item Специалист ИЛИ администратор щёлкает по нему для сортировки ИЛИ щёлкает ещё раз для изменения порядка сортировки.
		\item Отображаются отчёты в отсортированном виде.
	\end{itemize}
	Результат: Специалист ИЛИ администратор отсортировал отчёты по интересующему его признаку.
\end{itemize}

\begin{itemize}[topsep=0pt, parsep=0pt, itemsep=0pt, leftmargin=*, labelindent=0.5cm]
	\item \textit{Выполнение действия с отчётом} \\
	Действующее лицо: специалист ИЛИ администратор. \\
	Основной сценарий:
	\begin{itemize}[topsep=0pt, parsep=0pt, itemsep=0pt, leftmargin=*, labelindent=0.5cm]
		\item Специалист ИЛИ администратор открывает вкладку отчётов.
		\item В меню сверху специалист ИЛИ администратор выбирает действие.
		\item Специалист ИЛИ администратор выставляет флажки справа от тех отчётов, к которым он хотел бы применить это действие.
		\item Специалист ИЛИ администратор нажимает на кнопку \textquote{\en{go}}.
		\item Выбранное действие производится с выбранными отчётами.
	\end{itemize}
	Результат: Специалист ИЛИ администратор применил выбранное действие к выбранным отчётам.
\end{itemize}

\begin{itemize}[topsep=0pt, parsep=0pt, itemsep=0pt, leftmargin=*, labelindent=0.5cm]
	\item \textit{Редактирование отчёта} \\
	Действующее лицо: специалист ИЛИ администратор. \\
	Основной сценарий:
	\begin{itemize}[topsep=0pt, parsep=0pt, itemsep=0pt, leftmargin=*, labelindent=0.5cm]
		\item Специалист ИЛИ администратор открывает вкладку отчётов.
		\item Специалист ИЛИ администратор выбирает интересующий его отчёт.
		\item Специалист ИЛИ администратор нажимает на его дату отправки.
		\item Открывается окно редактирования отчёта.
		\item Специалист ИЛИ администратор редактирует информацию об отчёте.
		\item Специалист ИЛИ администратор нажимает кнопку \textquote{\en{Save}} или \textquote{\en{Save and continue editing}}.
		\item В зависимости от нажатой кнопки, открывается вкладка отчётов или окно редактирования отчёта.
	\end{itemize}
	Результат: Специалист ИЛИ администратор отредактировал информацию выбранного отчёта.
\end{itemize}

\begin{itemize}[topsep=0pt, parsep=0pt, itemsep=0pt, leftmargin=*, labelindent=0.5cm]
	\item \textit{Удаление отчёта} \\
	Действующее лицо: специалист ИЛИ администратор. \\
	Основной сценарий:
	\begin{itemize}[topsep=0pt, parsep=0pt, itemsep=0pt, leftmargin=*, labelindent=0.5cm]
		\item Специалист ИЛИ администратор открывает вкладку отчётов.
		\item Специалист ИЛИ администратор выбирает интересующий его отчёт.
		\item Специалист ИЛИ администратор нажимает на его дату отправки.
		\item Открывается окно редактирования отчёта.
		\item Специалист ИЛИ администратор нажимает кнопку \textquote{\en{Delete}}.
		\item Открывается вкладка отчётов.
	\end{itemize}
	Результат: Специалист ИЛИ администратор удалил выбранный отчёт.
\end{itemize}

\begin{itemize}[topsep=0pt, parsep=0pt, itemsep=0pt, leftmargin=*, labelindent=0.5cm]
	\item \textit{Поиск профиля пользователя по роли} \\
	Действующее лицо: администратор. \\
	Основной сценарий:
	\begin{itemize}[topsep=0pt, parsep=0pt, itemsep=0pt, leftmargin=*, labelindent=0.5cm]
		\item Администратор открывает вкладку профилей пользователей.
		\item В меню справа администратор настраивает фильтрацию профилей пользователей по роли.
		\item Отображается интересующая администратора выборка профилей пользователей.
	\end{itemize}
	Результат: Администратор отфильтровал профили пользователей системы по роли.
\end{itemize}

\begin{itemize}[topsep=0pt, parsep=0pt, itemsep=0pt, leftmargin=*, labelindent=0.5cm]
	\item \textit{Сортировка профилей пользователей} \\
	Действующее лицо: администратор. \\
	Основной сценарий:
	\begin{itemize}[topsep=0pt, parsep=0pt, itemsep=0pt, leftmargin=*, labelindent=0.5cm]
		\item Администратор открывает вкладку профилей пользователей.
		\item В меню сверху администратор выбирает поле профиля пользователя для сортировки.
		\item Администратор щёлкает по нему для сортировки ИЛИ щёлкает ещё раз для изменения порядка сортировки.
		\item Отображаются профили пользователей в отсортированном виде.
	\end{itemize}
	Результат: Администратор отсортировал профили пользователей по интересующему его признаку.
\end{itemize}

\begin{itemize}[topsep=0pt, parsep=0pt, itemsep=0pt, leftmargin=*, labelindent=0.5cm]
	\item \textit{Выполнение действия с профилем пользователя} \\
	Действующее лицо: администратор. \\
	Основной сценарий:
	\begin{itemize}[topsep=0pt, parsep=0pt, itemsep=0pt, leftmargin=*, labelindent=0.5cm]
		\item Администратор открывает вкладку профилей пользователей.
		\item В меню сверху администратор выбирает действие.
		\item Администратор выставляет флажки справа от тех профилей пользователей, к которым он хотел бы применить это действие.
		\item Администратор нажимает на кнопку \textquote{\en{go}}.
		\item Выбранное действие производится с выбранными профилями пользователей.
	\end{itemize}
	Результат: Администратор применил выбранное действие к выбранным профилям пользователей.
\end{itemize}

\begin{itemize}[topsep=0pt, parsep=0pt, itemsep=0pt, leftmargin=*, labelindent=0.5cm]
	\item \textit{Редактирование профиля пользователя} \\
	Действующее лицо: администратор. \\
	Основной сценарий:
	\begin{itemize}[topsep=0pt, parsep=0pt, itemsep=0pt, leftmargin=*, labelindent=0.5cm]
		\item Администратор открывает вкладку профилей пользователей.
		\item Администратор выбирает интересующий его профиль пользователя.
		\item Администратор нажимает на его адрес электронной почты.
		\item Открывается окно редактирования профиля пользователя.
		\item Администратор редактирует информацию о профиле пользователя.
		\item Администратор нажимает кнопку \textquote{\en{Save}} или \textquote{\en{Save and continue editing}}.
		\item В зависимости от нажатой кнопки, открывается вкладка профилей пользователей или окно редактирования профиля пользователя.
	\end{itemize}
	Результат: Администратор отредактировал информацию выбранного профиля пользователя.
\end{itemize}

\begin{itemize}[topsep=0pt, parsep=0pt, itemsep=0pt, leftmargin=*, labelindent=0.5cm]
	\item \textit{Блокирование профиля пользователя} \\
	Действующее лицо: администратор. \\
	Основной сценарий:
	\begin{itemize}[topsep=0pt, parsep=0pt, itemsep=0pt, leftmargin=*, labelindent=0.5cm]
		\item Администратор открывает вкладку профилей пользователей.
		\item Администратор выбирает интересующий его профиль пользователя.
		\item Администратор нажимает на его адрес электронной почты.
		\item Открывается окно редактирования профиля пользователя.
		\item Администратор снимает флажок поля \textquote{\en{Active}}.
		\item Администратор нажимает кнопку \textquote{\en{Save}} или \textquote{\en{Save and continue editing}}.
		\item В зависимости от нажатой кнопки, открывается вкладка профилей пользователей или окно редактирования профиля пользователя.
	\end{itemize}
	Результат: Администратор заблокировал выбранный профиль пользователя, данный пользователь, специалист или администратор больше не сможет войти в систему.
\end{itemize}

\begin{itemize}[topsep=0pt, parsep=0pt, itemsep=0pt, leftmargin=*, labelindent=0.5cm]
	\item \textit{Удаление отчёта} \\
	Действующее лицо: администратор. \\
	Основной сценарий:
	\begin{itemize}[topsep=0pt, parsep=0pt, itemsep=0pt, leftmargin=*, labelindent=0.5cm]
		\item Администратор открывает вкладку профилей пользователей.
		\item Администратор выбирает интересующий его профиль пользователя.
		\item Администратор нажимает на его адрес электронной почты.
		\item Открывается окно редактирования профиля пользователя.
		\item Администратор нажимает кнопку \textquote{\en{Delete}}.
		\item Открывается вкладка профилей пользователей.
	\end{itemize}
	Результат: Администратор удалил выбранный профиль пользователя, все связанные с ним отчёты сохранились.
\end{itemize}
