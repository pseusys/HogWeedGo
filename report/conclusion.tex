% !TeX spellcheck = russian-aot

\newcont{Заключение}

\tab 
В ходе выполнения работы были достигнутые поставленные цели: была разработана система, позволяющая решить проблему распространения потенциально опасных для человека видов растений и контроля их популяции путём автоматического учёта информации об их местонахождении, поступающей от широких масс добровольцев, и дающая группе экологов возможность осуществлять модерацию и анализ собранных данных.

\tab
В ходе разработки системы были решены следующий задачи:
\begin{enumerate}
	\item Была разработана архитектура системы, описаны её компоненты, модель БД и пути передачи данных.
	Также были разработаны сценарии использования системы и составлен макет пользовательского интерфейса.
	\item Было создано серверное приложение для сбора и хранения данных, а также системы анализа и модерирования наблюдений с web-интерфейсом.
	Оно было написано на языке программирования \en{Python} с использованием фреймворка \en{Django}.
	При помощи фреймворка \en{Django Rest Framework} для приложения был создан и описан API, с помощью платформы \en{Docker} настроены способы запуска и конфигурации.
	\item Было создано кроссплатформенное мобильное и web приложение, позволяющее добровольцем заносить наблюдения в систему.
	Оно было написано на языке \en{Dart} с использованием фреймворка \en{Flutter} и архитектуры \en{BLoC}.
\end{enumerate}

\tab
В результате тестирования серверной части приложения было выявлено увеличение времени, требующегося на обработку запросов отчётов при большом количестве запросов, загруженных в систему.
Несмотря на то, что на текущую реализацию клиентского приложения это оно влияния не оказывает, был предложен и реализован вариант решения данной проблемы.

\tab
В результате исследования последовательностей действий пользователей, необходимых для выполнения сценариев использования клиентского приложения, были выявлены и сокращены некоторые последовательности.

\tab
В случае появления необходимости наличия у системы дополнительных функций и возможностей, таких как, например, сортировка отчётов по удалённости от текущего местоположения пользователя или добавление новых методов авторизации пользователей (по номеру телефона или аккаунту в социальной сети), возможность их добавления была предусмотрена.
