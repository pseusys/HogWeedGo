% !TeX spellcheck = russian-aot

\newcont{Обозначения и сокращения}

\begin{itemize}
	\item Application Programming Interface (API) -- программный интерфейс приложения, описание способов, которыми одна программа имеет возможность взаимодействовать с другими.
	\item Business Logic Component (BLoC) -- шаблон проектирования, в основе которого лежит принцип обособления бизнес логики приложения от его графического интерфейса и связи их при помощи обработки потоков событий.
	\item Dart -- язык программирования, предназначенный для создания клиентских приложений.
	\item Django -- web фреймворк, разработанный на языке \en{Python}, предназначенный для создания web приложений при помощи языка \en{Python}, использующий шаблон проектирования \textquote{Модель-Представление-Контроллер}.
	\item Django Rest Framework (DRF) -- дополнительный набор инструментов для фреймворка \en{Django} для создания web API.
	\item Docker -- платформа для запуска приложений в изолированных виртуальных на уровне операционной среды средах, называемых контейнерами.
	\item docker-compose -- программа для настройки и запуска систем, состоящих из большого количества \en{Docker} контейнеров.
	\item Flutter -- фреймворк для языка программирования \en{Dart}, позволяющий разрабатывать с его помощью кросс-платформенные приложения.
	\item PostgreSQL -- объектно-реляционная система управления базами данных.
	\item PostGIS -- программа, добавляющая поддержку географических объектов в реляционную базу данных PostgreSQL.
	\item Python -- высокоуровневый интерпретируемый язык программирования общего назначения.
	\item TensorFlow -- программная библиотека для машинного обучения, построения и тренировки нейронных сетей.
	\item Фреймворк -- программное обеспечение, облегчающее разработку и объединение разных компонентов большого программного проекта.
\end{itemize}
