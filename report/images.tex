% !TeX spellcheck = russian-aot

\newcommand{\pictureone}{
	\begin{figure}[!htbp]
		\fontsize{10}{12}
		\centering
		\begin{tikzpicture}[thick, on grid]
			% Server
			\node [block] (server) {Сервер};
			% Database
			\node [block, right= of server] (bd) {База данных};
			\draw[<->] (server) -- node[link, above] {Данные о растениях} (bd); 
			% Receiver API
			\node [block, below left= of server] (ext-api) {API для получения опубликованных данных};
			\draw[->] (server) -- node[link, below right] {Обработанные данные} (ext-api);
			
			% Service site
			\node [block, above= of server] (service-site) {Сервисный сайт};
			\draw[<->] (server) -- node[link, right] {Модерация и изменение статуса данных} (service-site);
			% Public site
			\node [block, above left= of server] (public-site) {Сайт с опубликованными данными};
			\draw[->] (ext-api) -- node[link, right] {Сортированные данные} (public-site);
			
			% Sender API
			\node [block, below right= of server] (int-api) {API для получения данных};
			\draw[<-] (server) -- node[link, below left] {Данные, полученные от пользователей} (int-api);
			% WEB client
			\node [block, below= of ext-api] (web-cli) {WEB клиент};
			\draw[<-] (int-api) edge[bend left=35, looseness=1.8] node[link, below right] {Данные о найденном растении} (web-cli);
			\draw[->] (ext-api) -- node[link, left] {Cортированные данные} (web-cli);
			% Android client
			\node [block, right= of web-cli] (and-cli) {Android клиент};
			\draw[<-] (int-api) -- node[link, above left] {Данные о найденном растении} (and-cli);
			\draw[->] (ext-api) -- node[link, above right] {Сортированные данные} (and-cli);
			% iOS client
			\node [block, right= of and-cli] (ios-cli) {iOS клиент};
			\draw[<-] (int-api) -- node[link, right] {Данные о найденном растении} (ios-cli);
			\draw[->] (ext-api) edge[bend right=35, looseness=1.8] node[link, below left] {Сортированные данные} (ios-cli);
		\end{tikzpicture}
		\caption{Схема архитектуры}
	\end{figure}
}

\newcommand{\picturetwo}{
	\begin{figure}[!htbp]
		\fontsize{10}{12}
		\centering
		\begin{tikzpicture}[thick, node distance = 9cm and 5cm, on grid]
			% User
			\node [block, text width=7.5cm, align=left] (user) {
				\centering{\textbf{Пользователь}}
				\begin{itemize}[parsep=0pt, leftmargin=*, labelindent=0cm]
					\item \underline{Адрес электронной почты}
					\item Хэш сумма пароля
					\item Имя
					\item Дата регистрации в системе
					\item Время последнего входа в систему
					\item Является ли пользователь администратором
					\item Является ли пользователь специалистом
					\item Активен ли аккаунт пользователя
					\item Фото профиля
					\item Уменьшенное фото профиля
				\end{itemize}
			};
			% Report
			\node [block, text width=4.5cm, align=left, below right= of user] (report) {
				\centering{\textbf{Отчёт}}
				\begin{itemize}[parsep=0pt, leftmargin=*, labelindent=0cm]
					\item \underline{Номер отчёта}
					\item Дата и время наблюдения
					\item Адрес
					\item Первый комментарий
					\item Координаты
					\item Статус
					\item Вид растения
				\end{itemize}
			};
			\draw[<-] (user) edge ($(user)!0.55!(report)$) edge[dashed] (report);
			% Comment
			\node [block, text width=4cm, align=left, below left= of user] (comment) {
				\centering{\textbf{Комментарий}}
				\begin{itemize}[parsep=0pt, leftmargin=*, labelindent=0cm]
					\item Текст комментария
				\end{itemize}
			};
			\draw[<-] (user) edge ($(user)!0.65!(comment)$) edge[dashed] (comment);
			\draw[<-] (report) edge (comment);
		\end{tikzpicture}
		\caption{ER диаграмма модели}
	\end{figure}
}
