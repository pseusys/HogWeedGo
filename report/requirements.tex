% !TeX spellcheck = russian-aot

\section{Формулировка требований и постановка задачи}

\subsection{Постановка задачи}

\tab
Необходимо разработать систему, позволяющую осуществлять наблюдение за распространением потенциально опасных для человека растений. \\

\subsection{Формулировка требований}

\tab
Такая система должна соответствовать всем критериям, использованным для сравнения аналогов. \\

\subsection{Выбор метода решения}

\tab
В качестве источника информации и средства наблюдения предлагается использовать неограниченное количество волонтёров, заинтересованных в решении экологических проблем.
Эти люди в данной работе будут называться \linkterm{users}{пользователями} системы.
Такое решение позволит получать большое количество актуальной информации, но в следствие отсутствия профессиональной подготовки волонтёров точность такой информации снизится.
Также возможно возникновение случаев необходимости модерации присланных волонтёрами данных.

\nwln
В качестве меры модерации предлагается необходимость регистрации волонтёров при помощи подтверждённого адреса электронной почты (в последствии возможно добавление других методов авторизации), а также наличие возможности редактирования и удаления всех загруженных данных.
Кроме того, для предоставления возможности связи с волонтёрами, в каждый отчёт необходимо включить информацию о пользователе, совершившем наблюдение, описанное в отчёте.
Люди, в обязанности которых будет входить модерация отчётов и взаимодействие с пользователями в данной работе будут называться \linkterm{admins}{администраторами} системы.

\nwln
В качестве меры повышения точности присланных волонтёрами данных предлагается создание автоматической системы распознавания определённых видов растений на фотографиях, а также ручная проверка присланных пользователями данных.
Желательно повысить точность предварительной автоматической фильтрации фотографий до 90\% или выше, при этом размер модели нейронной сети не должен превышать 100 mb, что позволит использование её в клиентских приложениях.
Люди, в обязанности которых будет входить проверка и работа с присланными данными в данной работе будут называться \linkterm{specialists}{специалистами} системы.

\nwln
Для охвата более широкой группы добровольцев и их удобства необходима возможность отложенной отправки данных в случае, если пользователь находится вне зоны действия сети.
Этого возможно добиться при помощи создания мобильных приложений для операционных систем iOS и \en{Android}.
Также необходимо наличие web интерфейса отправки данных для пользователей, не желающих устанавливать приложение.
Кроме того желательно предоставление возможности интеграции системы со сторонними клиентскими приложениями путём создания документированного API для отправки данных в систему.

\nwln
Для осуществления модерации и работы с данными требуется создание сервисного web интерфейса, который далее в данной работе будет называться \linkterm{servicesite}{сервисный сайт} системы.
Также для удобства использования собранных данных в полевых условиях (например, во время поиска и обезвреживания растения), а также сбора статистики необходимо наличие способа выгрузки и сохранения обработанных данных в формате \textquote{CSV}.
Также желательно наличие возможности демонстрации статистики собранных данных на сторонних сайтах путём создания документированного API для демонстрации данных, загруженных в систему. \\

\subsection{Выбор технологий и архитектуры}

\tab
Во ходе выбора технологий особое внимание было уделено лицензии, под которой распространяется их исходный код, так как распространение конечного продукта планируется под открытой лицензией.

\nwln
Для реализации сервиса была выбрана клиент-серверная архитектура. Схема архитектуры представлена на рисунке 1. \\
\pictureone

\tab
Серверная часть системы была написана на языке программирования \en{Python} с использованием фреймворка \en{Django}.
Такой выбор обусловлен тем, что этот фреймворк с одной стороны предлагает единый интерфейс доступа к различным базам данных, что будет полезно в случае необходимости изменения СУБД, а с другой стороны удовлетворяет необходимость создания графического пользовательского интерфейса для администраторов приложения, автоматически генерируя сервисный сайт.

\nwln
В качестве системы управления базами данных была использована \en{PostgreSQL}, так как она распространяется под открытой лицензией, а также имеет специальные геометрические типы данных, которые можно использовать для работы с географическими координатами.
Для работы с ними также используется программа \en{PostGIS}.
Она предоставляет увеличение скорости работы до 2.5 раз по сравнению с аналогами (например, \en{MongoDB}) при работе со сложными запросами к геометрическим данным (например, вычислений расстояния между точками)\cite{postgis-vs-mongo}.

\nwln
Код приложения-клиента был написан на языке \en{Dart} с использованием фреймворка \en{Flutter}\cite{flutter-vs-react}.
Такой выбор обусловлен необходимостью одновременной разработки для мобильных платформ и web браузеров.
Разработка единой кодовой базы для всех платформ позволяет сократить время разработки, а также бюджет и расходы на поддержку проекта.
Фреймворк \en{Flutter} распространяется под открытой лицензией, а также позволяет создать одинаковый пользовательский интерфейс для всех платформ и добиться высокой производительности на каждой из них, что улучшает опыт пользователя\cite{flutter-better}.

\nwln
Для хранения и передачи пользовательских данных в клиентском приложении была выбрана архитектура BLoC\cite{bloc-better}, специально разработанная для этого фреймворка, и позволяющая отделить бизнес-логику программы от пользовательского интерфейса, что в свою очередь приведёт к меньшим временным и финансовым затратам на тестирование программы.

\nwln
В качестве средства повышения точности присланных пользователями данных была выбрана свёрточная нейронная сеть\cite{convolutional-better}, так как она позволяет достаточно точно произвести классификацию фотографий.
Для реализации нейронной сети был выбран фреймворк \en{TensorFlow}\cite{tensorflow-better}, так как он распространяется под открытой лицензией, а также позволяет достичь достаточно высоких показателей точности классификации изображений.
В качестве модели для применения технологии \en{transfer learning} была выбрана модель \en{XCeption}\cite{xception-better}, так как с её помощью можно достичь необходимой точности классификации, при этом не превысив требуемое ограничение по размеру.