% !TeX spellcheck = russian-aot

\section{Описание решения}

В соответствии с описанной архитектурой были разработаны следующие компоненты системы:

\subsection{Серверная часть ПО}
\label{subsec:serverside-app}

\tab
Серверная часть системы была написана на языке программирования \en{Python} с использованием фреймворка \en{Django}.
Такой выбор обусловлен тем, что этот фреймворк с одной стороны предлагает единый интерфейс доступа к различным базам данных, что будет полезно в случае необходимости изменения СУБД, а с другой стороны удовлетворяет необходимость создания графического пользовательского интерфейса для администраторов приложения, автоматически генерируя сервисный сайт.

\tab
В качестве системы управления базами данных была использована \en{PostgreSQL}, так как она распространяется под открытой лицензией, а также имеет специальные геометрические типы данных, которые можно использовать для работы с географическими координатами.
Для работы с ними также используется программа \en{PostGIS}.
Она предоставляет увеличение скорости работы до 2.5 раз по сравнению с аналогами (например, \en{MongoDB}) при работе со сложными запросами к геометрическим данным (например, вычислений расстояния между точками)\cite{postgis-vs-mongo}.

\tab
Для каждого отношения в базе данных был создан класс-модель, наследник класса \code{Model} фреймворка \en{Django}.
Эти классы позволяют программе получать доступ к сохранённым в БД данным.
В этих классах на значения некоторых полей БД были наложены ограничения, представленные в таблице 2.
\tabletwo

\tab
Также для каждого отношения в базе данных был создан класс-админ, наследник класса \code{ModelAdmin} фреймворка \en{Django}.
Эти классы управляют представлением данных из БД на сервисном сайте системы, описывают порядок и вид отображения полей, а также значения каких полей могут быть изменены администраторами системы.

\tab
Кроме того были описаны действия, которые можно производить с несколькими объектами из базы данных одновременно, они описаны в таблице 3.
\tablethree

\tab
Были добавлены необходимые для работы фреймворка \en{Django} классы, изменён внешний вид отображения нескольких полей (например, поля пароля), добавлены служебные функции (например, для создания и сохранения уменьшенной копии фотографий) и константы.

\tab
Для импорта и экспорта данных, хранящихся в системе, была добавлена консольная команда, описанная в классе, наследнике класса \code{BaseCommand} фреймворка \en{Django}.
Эта команда позволяет загружать в систему данные, хранящиеся в JSON файле (не сжатом или сжатом при помощи алгоритма \en{Deflate}), выгружать из неё данные в такой же файл, а также очищать базу данных системы и удалять загруженные пользователями изображения.

\subsection{API серверной части ПО}

\tab
В качестве архитектуры API серверной части ПО была выбрана архитектура REST\cite{rest-better}, так как она позволяет кеширование данных, полученных от серверного приложения, что уменьшает нагрузку на интернет соединение пользователя в случае большого количества отчётов, а также позволяет серверному приложению не хранить данные о состоянии пользователя, что увеличивает возможности масштабирования системы.

\tab
Для создания API был использован \en{Django REST Framework}, позволяющий создавать интерфейсы доступа к приложениям, написанным при помощи Django, так как он поддерживает множество видов аутентификации пользователей, а также позволяет настроить способы сериализации и десериализации данных при отправке их пользователям и получении от пользователей соответственно.

\tab
Для каждого отношения в базе данных был создан класс-сериализатор, наследник класса \code{ModelSerializer} фреймворка \en{DRF}.
Эти классы позволяют корректно переводить данные из Python объектов, которые фреймворк \en{Django} может загружать в базу данных, в JSON объекты и обратно.
Также эти классы обрабатывают и сохраняют присланные пользователями фотографии.
Каждый из этих классов может работать в нескольких режимах, преобразуя данные разным образом для отправки пользователям, для выгрузки из системы в виде файла и, в случае отчётов, для выгрузки в виде .CSV таблицы.
В таблице 4 описано, как преобразуются поля разных объектов при обработке сериализаторами в разных режимах работы (S обозначает сериализацию, а D -- десериализацию).
\tablefour

\tab
Также для каждого отношения в базе данных, кроме Фотографии, был создан класс-представление, наследник класса \code{GenericViewSet} фреймворка \en{DRF}.
Эти классы позволяют пользователям получать доступ к данным, сохранённым в системе, при помощи HTTP запросов.
Для отношения Фотография класса-представления создано не было, так как при сериализации фотографии вкладываются в отчёты, к которым они приложены.
Описание данных, которыми серверная часть системы обменивается с администрацией и пользователями при помощи API представлено на рисунке 4.
\picturefour

\tab
Для отправки данных в систему пользователям необходима авторизация. В качестве метода авторизации был выбран метод авторизации при помощи токена. При регистрации аккаунта токен сохраняется в базе данных, после чего, при каждом вызове методов API, требующих авторизации, пользователь передаёт свой токен в заголовке \code{Authorization}. При удалении аккаунта токен из базы данных также удаляется.

\tab
Для обеспечения возможности авторизации был создан отдельный класс-представление, позволяющий пользователям создавать профиль, входить в систему, подтверждать адрес электронной почты, менять данные своего профиля, выходить из системы и удалять свой профиль.
Кроме того, для проверки доступности и быстродействия системы был добавлен ещё один HTTP запрос, \code{healthcheck}.

\tab
Так как в соответствии с REST архитектурой серверное ПО не должно хранить данные о состоянии пользователя, для подтверждения адреса электронной почты пользователя используется следующий код, состоящий из восьми букв латинского алфавита и цифр, созданный при помощи функции \code{random.choice}.
В качестве ключа для этой функции выступает конкатенация адреса электронной почты пользователя, \en{DJANGO\-\_SECRET\-\_KEY} (уникального для каждого работающего экземпляра программы ключа) и \en{Unix Timestamp} текущего времени, делённого без остатка на 600, т. е. округлённого до 10 минут.
Таким образом обеспечивается как проверка адреса электронной почты, так и ограничение на действие кода в 10 минут.

\tab
Для предотвращения DDoS атак на систему было предусмотрено ограничение на отправку пользователем требующих авторизации запросов в 4 запроса в минуту.

\tab
HTTP запросы пользователей обрабатываются при помощи объекта класса \code{Simple\-Router}, входящего в состав фреймворка \en{DRF}. \en{OpenAPI} спецификация API системы представлена в \additionref{openapi}.

\subsection{Клиентская часть ПО}

\tab
Код приложения-клиента был написан на языке \en{Dart} с использованием фреймворка \en{Flutter}\cite{flutter-vs-react}.
Такой выбор обусловлен необходимостью одновременной разработки для мобильных платформ и web браузеров.
Разработка единой кодовой базы для всех платформ позволяет сократить время разработки, а также бюджет и расходы на поддержку проекта.
Фреймворк \en{Flutter} распространяется под открытой лицензией (чему было уделено особое внимание, так как распространение конечного продукта планируется под открытой лицензией), а также позволяет создать одинаковый пользовательский интерфейс для всех платформ и добиться высокой производительности на каждой из них, что улучшает опыт пользователя\cite{flutter-better}.

\tab
Для хранения и передачи пользовательских данных в клиентском приложении была выбрана архитектура BLoC\cite{bloc-better}, специально разработанная для этого фреймворка, и позволяющая отделить бизнес-логику программы от пользовательского интерфейса, что в свою очередь приведёт к меньшим временным и финансовым затратам на тестирование программы.

\tab
Для каждого отношения в базе данных был создан класс-модель, наследник класса \code{Equatable} библиотеки \en{equatable}.
При помощи библиотек \en{json\-\_serializable} и \en{json\-\_annotation} для каждого из этих классов были сгенерированы функции сериализации и десериализации, которые при отправке данных в систему исключают лишние и пустые поля.

\tab
Были созданы классы бизнес логики, наследники класса \code{Bloc} библиотеки \en{bloc}.
Они были разделены в соответствии с макетом пользовательского интерфейса так, чтобы с каждым экраном, с которым взаимодействует пользователь, был связан отдельный класс бизнес логики.
Исключение представляет класс \code{Account\-Bloc}, который связан одновременно с экраном авторизации и изменения данных профиля пользователя.
Эти классы позволяют реагировать на сигналы, посланные пользователем, устройством пользователя или серверной частью системы.
В качестве реакции может выступать изменение его текущего состояния, HTTP запрос на серверную часть системы и т. д.

\tab
Для каждого класса бизнес логики были созданы классы сигналов и состояний.
Классы сигналов содержат информацию о том, какой сигнал поступил в обработку, а также данные, входящие в состав этого сигнала.
Классы состояния хранят текущее состояние одной конкретной части бизнес логики программы и позволяют пользовательскому интерфейсу реагировать на его изменения.
Также для некоторых классов бизнес логики были созданы классы форм, наследники класса \code{FormzInput} библиотеки formz.
Эти классы позволяют проверять корректность данных введённых в различные поля для ввода пользовательского интерфейса перед их отправкой.

\tab
Для каждого класса бизнес логики были созданы классы репозиториев.
Такие классы содержат функции, отправляющие HTTP запросы при помощи API серверной части системы, а также функции обработки ответов и ошибок.
Один из них, класс \code{Account\-Repository}, хранит также токен авторизации текущего пользователя (при его наличии).
При каждом изменении токен сохраняется в локальное хранилище браузера при помощи библиотеки \code{shared\-\_preferences} и считывается оттуда при перезагрузке приложения.

\tab
При помощи стандартных виджетов, входящих в состав фреймворка \en{Flutter} был реализован пользовательский интерфейс в соответствии с спроектированным макетом.
В интерфейс заложена возможность изменения при изменении состояния бизнес логики приложения.

\subsection{Система автоматической проверки изображений}

\tab
В качестве средства повышения точности присланных пользователями данных была выбрана свёрточная нейронная сеть\cite{convolutional-better}, так как она позволяет достаточно точно произвести классификацию фотографий.
Для реализации нейронной сети был выбран фреймворк \en{TensorFlow}\cite{tensorflow-better}, так как он распространяется под открытой лицензией, а также позволяет достичь достаточно высоких показателей точности классификации изображений.

\tab
В качестве модели для применения технологии \en{transfer learning} была выбрана модель \en{MobileNetV2}\cite{mobilenet-better}, обученная на наборе данных \en{Image\-Net}, так как с её помощью можно достичь необходимой точности классификации, при этом не превысив требуемое ограничение по размеру.

\tab
Система автоматической проверки изображений -- желательный, но не обязательный компонент системы.
Для каждой конкретной ситуации наблюдения за определённым набором видов растений модель необходимо обучать заново, используя при этом фотографии тех растений, за которыми производится наблюдение.
В качестве примера того, как именно она может быть обучена, был написан блокнот \en{JuPyter Notebook}, позволяющий обучить модель для определения борщевика Сосновского.
Для обучения нейронной сети определению других видов растений, его необходимо частично изменить. 

\tab
Так как система автоматической проверки изображений используется также для того, чтобы подсказывать пользователю, каким образом он может улучшить свою фотографию, в её задачи входит разделение всех фотографий на три класса: \en{hogweed} (фотографии, на которых изображён борщевик Сосновского), \en{cetera} (фотографии, на которых изображены другие растения) и \en{other} (фотографии, на которых не изображены растения).
Изображения, на которых не изображены растения были взяты из набора данных \en{Open\-Images}\cite{openimage-better}, тогда как изображения борщевика и других растений были выгружены из системы \en{iNaturalist}\cite{inaturalist}.
В общем при обучении использовалось по 7100 фотографий каждого типа.

\tab
Перед началом обучения изображения были скачаны по ссылкам, взятым из .CSV файлов наборов данных.
Для избежания возможных погрешностей работы сети, вызванных обучением на нескольких одинаковых изображениях, во время скачивания для каждого изображения был рассчитан хэш и, в случае совпадения этого хэша с хэшем какого-либо изображения, загруженного ранее, оно пропускалось.

\tab
Изображения были разделены на набор для обучения и набор для валидации, после чего оба набора были подвергнуты аугментации при помощи случайного горизонтального поворота изображений и случайного угла наклона не более чем на 18 градусов.
Для соответствия критериям ввода модели \en{MobileNetV2} изображения были уменьшены до размера 224 на 224 пикселей.

\tab
Модель была дополнена новыми слоями и скомпилирована.
В качестве алгоритма оптимизации был использован алгоритм \en{Adam}, а в качестве функции потерь -- функция категориальной кросс-энтропии.
Прогресс обучения измерялся по точности классификации.
Схема скомпилированной модели представлена на рисунке 5.
\picturefive

\tab
Обучение верхних слоёв модели производилось в течении 16 эпох, при этом оно прекращалось как только прогресс за две эпохи подряд не превосходил 0.01 процента.
Темп обучения составил 0.0001.
График обучения представлен на рисунке 6.
\picturesix

\tab
Дообучение модели (обучение с 80\% \textquote{размороженных} нижних, предобученных слоёв) производилось в таких же условиях, только на этот раз в качестве алгоритма оптимизации был использован алгоритм \en{RMSprop}, а темп обучения составил 0.00001.
График обучения представлен на рисунке 7.
\pictureseven

\tab
После окончания обучения модель была преобразована в формат .TFLITE для использования в клиентской части системы.

\subsection{Процесс установки и запуска системы}
\label{subsec:installing-and-launching}

\tab
Вследствие того, что серверная часть системы требует установки и настройки сразу нескольких программ определённых версий, было принято решение автоматизировать процесс её запуска.

\tab
Пользовательские настройки системы, такие как \en{DJANGO\-\_SECRET\-\_KEY}, доменное имя сервера, имя пользователя и пароль администратора системы и базы данных, хранятся в конфигурационном файле в корневой директории серверной части системы.
Настройки из этого файла передаются в систему в виде переменных среды.
Также в ней в специальной папке с сертификатами хранятся сертификаты для запуска серверной части системы с возможностью доступа при помощи протокола HTTPS и подписи отправляемых системой электронных писем.
Для их генерации был написан скрипт, в качестве параметров принимающий настройки, которые пользователь желает задать сам, и генерирующий остальные.

\tab
В систему был добавлен отладочный режим работы.
От стандартного он отличается тем, что в нём по умолчанию используется HTTP вместо HTTPS, демонстрируются развёрнутые сообщения об ошибках, а также вместо электронных писем коды авторизации доставляются пользователям в теле ответов на запросы подтверждения электронной почты.
Отладочный режим работы используется по умолчанию при локальном запуске системы.

\tab
Для упрощения локального запуска системы был написан скрипт локального запуска.
В его задачи не входит полная настройка системы, так как она включает в себя действия, потенциально небезопасные для устройства пользователя, такие как, например, установка определённой версии СУБД от имени администратора.
Скрипт упрощения локального запуска проверяет соответствие системы требованиям, выводит на экран советы по установке программ и выполняет их настройку.
Также с его помощью можно установить локальные переменные, необходимые для запуска серверной части системы.
Локальный режим запуска системы не рекомендуется использовать при стандартном режиме работы из-за уязвимостей системы, которые могут возникнуть при некорректной настройке.

\tab
В качестве альтернативы локальному режиму запуска предлагается использовать режим запуска, использующий платформу \en{Docker} и утилиту \en{docker-compose}.
Эта платформа позволяет разворачивать программы в виртуальных окружениях, называемых контейнерами, использующих возможности виртуализации операционной системы \en{Linux}.
Для его реализации был создан файл с определениями версий и настроек контейнеров, используемых системой, а также файл, содержащий команды для сборки образа самой системы.
Для связи контейнеров используется виртуальная сеть, а также общие для нескольких контейнеров тома.
В этом режиме запуска системе не требуется никаких дополнительных настроек кроме конфигурационного файла.
Также этот режим по умолчанию использует сервер \en{Daphne} и стандартный режим запуска системы.

\tab
Контейнер, содержащий серверную часть системы был опубликован в регистре контейнеров \en{GitHub}\cite{container-register}, что позволило реализовать упрощённый режим запуска системы, для которого не требуется ничего кроме конфигурационных файлов системы и файла с описанием настроек контейнеров \code{docker-compose.yml}.

\tab
Автоматизация сборки клиентской части системы затруднительна, так как одной из основных констант в ней является адрес сервера, на котором запущена серверная часть системы, и этот адрес может меняться от сборки к сборке.
Тем не менее, чтобы избежать необходимости вносить изменения в её исходный код, эта константа была вынесена в параметры сборки и определена при помощи \code{dart-define}.

\subsection{Непрерывная интеграция и развёртывание}

\tab
Для удобства тестирования и автоматизации сборки и публикации артефактов системы, таких как, например, \en{Docker} контейнер, содержащий её серверную часть, была использована платформа \en{GitHub Actions}.

\tab
С её помощью после каждого обновления исходного кода системы выполняются следующие сценарии:
\begin{itemize}
	\item Если был изменён исходный код серверной части системы:
	\begin{enumerate}
		\item Производится сборка и локальный запуск системы в отладочном режиме работы, проводятся unit тесты.
		\item Производится сборка и запуск системы с использованием платформы \en{Docker} в отладочном режиме работы, проводятся тесты API при помощи программы \en{Postman} и библиотеки \en{newman}.
		\item Если пункты 1 и 2 были выполнены успешно, собирается и публикуется \en{Docker} контейнер, содержащий новую версию серверной части системы.
		\item Если пункт 3 был выполнен успешно, собирается архив, содержащий все необходимые скрипты и конфигурационные файлы для запуска системы в упрощённом режиме и публикуется на странице системы в \en{GitHub}\cite{project-repo}.
	\end{enumerate}
	\item Если был изменён исходный код клиентской части системы:
	\begin{enumerate}
		\item Собирается web версия клиентского приложения и публикуется на странице системы в \en{GitHub Pages}\cite{published-version}.
	\end{enumerate}
	Это действие также можно выполнить вручную со страницы системы в \en{GitHub}, указав при этом адрес сервера для клиентской части системы (по умолчанию используется \code{localhost}).
	\item Если был изменён исходный код системы автоматической проверки изображений:
	\begin{enumerate}
		\item Производится тестирование последней версии модели системы автоматической проверки изображений, опубликованной в разделе \en{Releases}.
	\end{enumerate}
\end{itemize}

\tab
Использование облачной платформы для сборки и развёртки системы не только помогает отслеживать появление ошибок после публикации каждого обновления исходного кода системы, но и даёт возможность пользователям системы отказаться от её самостоятельной настройки в пользу использования готовых и автоматически собранных артефактов.
