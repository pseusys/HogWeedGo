% !TeX spellcheck = russian-aot

\section{Описание решения}

\subsection{Сценарии использования}

\tab
Были разработаны следующие сценарии использования решения:

\begin{itemize}[topsep=0pt, parsep=0pt, itemsep=0pt, wide=0.5cm]
	\item \textit{Просмотр пользователем опубликованной отсортированной информации о наблюдениях} \\
	Действующее лицо: пользователь. \\
	Основной сценарий:
	\begin{itemize}[topsep=0pt, parsep=0pt, itemsep=0pt, wide=0.5cm]
		\item Пользователь открывает приложение клиент.
		\item Пользователь нажимает на один из указателей на карте, ему демонстрируются подробные сведения об отчёте и комментарии к нему.
		\item По желанию пользователь нажимает на фотографии, прикреплённые к отчёту и просматривает их увеличенную версию.
	\end{itemize}
	Результат: Пользователь ознакомился как с общей статистикой распространения потенциально опасных растений на карте, так и с заинтересовавшими его конкретными отчётами.
	
	\item \textit{Просмотр пользователем информации о системе, её задачах и краткого руководства по использованию} \\
	Действующее лицо: пользователь. \\
	Основной сценарий:
	\begin{itemize}[topsep=0pt, parsep=0pt, itemsep=0pt, wide=0.5cm]
		\item На основной странице приложения пользователь открывает меню навигации и выбирает пункт \textquote{\en{About}}, ему демонстрируется информация о системе, её задачах и краткое руководство по использованию.
	\end{itemize}
	Результат: Пользователь посмотрел данные со всей информацией по операциям конкретного изделия.
	
	\item \textit{Авторизация пользователя в системе} \\
	Действующее лицо: пользователь. \\
	Условие: На основной странице приложения пользователь открывает меню навигации и выбирает пункт \textquote{\en{Account}} ИЛИ на основной странице приложения нажимает на кнопку справа внизу с символом \textquote{+}. \\
	Основной сценарий:
	\begin{itemize}[topsep=0pt, parsep=0pt, itemsep=0pt, wide=0.5cm]
		\item Открывается страница авторизации пользователя.
	\end{itemize}
	Результат: Пользователь авторизован в системе и может выполнять требующие авторизации сценарии.
\end{itemize}

\subsection{Пользовательский интерфейс}

\subsection{Модель базы данных}

\subsection{Серверная часть ПО}

\subsection{API серверной части ПО}

\subsection{Клиентская часть ПО}

\subsection{Система автоматической проверки изображений}

\subsection{Процесс установки и запуска системы}

\subsection{Непрерывная интеграция и развёртывание}
